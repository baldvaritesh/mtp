\chapter{Utility}

\section{Introduction}
         
Here, we explain some related funtions which are used by stated anomaly 
detection techniques. Note that some of these functions can be used to process 
the results of the anomaly techniques.

\subsection{Functions used by Anomaly detection techniques}

\begin{itemize}
 \item convertListToFloat(li) \\
	Given list of elements, this function type casts all the elements to 
float type.
	
	\begin{itemize}
	  \item Input Parameters
	  
	  \begin{enumerate}
	    \item li \textit{(int)} : list of elements	    
	  \end{enumerate}

	  \item Output \textit{(list)}: \\
	  Returns list of elements after converting each element into float 

	  \end{itemize}
    
	
	
	
	
 \item getColumnFromListOfTuples(lstTuples,i) \\
 
    This function returns i'th element of all tuples as a list.
    
    \begin{itemize}
	  \item Input Parameters
	  
	  \begin{enumerate}
	    \item lstTuples \textit{(int)} : list of tuples. Tuples has no fixed 
format
	    \item i \textit{(int)} : index of which tuple element to return, 
index starting from zero
	  \end{enumerate}

	  \item Output \textit{(list)}: \\
	  Returns list of elements after fetching i'th element from each tuple.
	  
    \end{itemize}
 
 
 \item findAverageTimeSeries(timeSeriesCollection) \\
 
	It takes 2D list of element, where each element of timeSeriesCollection 
is one time series. It returns average of all time series. (first element of 
resltant time series will be average of first element of all time series) \\
	
	For example let,\\
	timeSeriesCollection: [ \newline
	    [1,2,3], \# Timeseries 1 \newline
	    [4,5,6], \# Timeseries 2 \newline
	    [7,8,9] \# Timeseries 3 \newline
	] \newline
	\\
	This function will return,\newline
	[4,5,6] \newline
	
	\begin{itemize}
	  \item Input Parameters
	  
	  \begin{enumerate}
	    \item timeSeriesCollection \textit{(list)} : 2D array of float 
elements.
	  \end{enumerate}

	  \item Output \textit{(list)}: \\
	  Returns list after taking average of all time series.
	  
	\end{itemize}
 
 
 
 \item writeToCSV(lstData,fileName) \\
 \item concateLists(lstData) \\
 \item cleanArray(array) \\
 
	This function removes ``nan'' (Not a number values) from the list.
	
	\begin{itemize}
	  \item Input Parameters
	  
	  \begin{enumerate}
	    \item array \textit{(list)} : list of float type elements
elements.
	  \end{enumerate}

	  \item Output \textit{(list)}: \\
	  Returns list of float elements after removing ``nan'' elements.
	  
	\end{itemize}
 
 
 \item MADThreshold(array) \\
 \item smoothArray(array, alpha = 2.0/15.0) \\
 \item csv2array(filePath) \\
 \item getColumn(array, column\_number) \\
 \item formatCSV2Array(z) \\

 \item csvTransform(filePath,startDate) \\

\end{itemize}


\subsection{Functions used to process results}

\begin{itemize}
   
   
   \item intersection(numOfResults, list1, resultOf1, list2, resultOf2, list3 = 
[], resultOf3="linear\_regression", list4=[], resultOf4="graph\_based", 
list5=[], resultOf5="spike\_detection" , list6=[], resultOf6="multiple\_arima") 
\\
  This function is used to take intersection of results of multiple methods (2 
or more), which are passed as list here. This function requires minimum of 5 
arguments. 
   
   \begin{itemize}
 \item Input Parameters
 
 \begin{enumerate}
  \item numOfResults \textit{(int)} : number of lists you are passing, minimum 2
  \item $list_i$ \textit{(list)} : list represnting result of the i'th 
algorithm. Note that this is list of tuples of the format, \\
  (startDate, endDate, value)
  \item $resultOf_i$ \textit{(string)}: this variable states, $list_i$ is of 
which algorithm, it can be from following: \\
            (slope\_based, linear\_regression, graph\_based, spike\_detection, 
multiple\_arima)

 \end{enumerate}

 \item Output \textit{(list)}: \\
 This function returns intersection of all lists. Returned value is list of 
tuples of the form: \\
 (date, correlation value, slope\_based value, linear\_regression value, 
graph\_based value, spike\_detection value, multiple\_arima value)

 \end{itemize}
   
   
   
   
   \item intersectionOfFinalResults(list1, list2) \\
  
  This function takes intersection of 2 lists (where each list is list of 
tuple) and returns result. Note that these input lists are generated as a output 
of ``intersection'' method.
 \begin{itemize}
 \item Input Parameters
 
 \begin{enumerate}
  \item list1 \textit{(list)} : List of tuples of the format: \\
  (date, correlation value, slope\_based value, linear\_regression value, 
graph\_based value, spike\_detection value, multiple\_arima value)
  \item list2 \textit{(list)} :  List of tuples of the format:  \\
  (date, correlation value, slope\_based value, linear\_regression value, 
graph\_based value, spike\_detection value, multiple\_arima value)

 \end{enumerate}

 \item Output \textit{(list)}: \\
 This function returns intersection of list1 and list2. Returned value is list 
of tuples of the form: \\
 (date, correlation value, slope\_based value, linear\_regression value, 
graph\_based value, spike\_detection value, multiple\_arima value)

 \end{itemize}
 
 
 
 \item unionOfFinalResults(list1, list2) \\
 This function takes union of 2 lists (where each list is list of tuple) and 
returns result. Note that these input lists are generated as a output of 
``intersection'' method.
 \begin{itemize}
 \item Input Parameters
 
 \begin{enumerate}
  \item list1 \textit{(list)} : List of tuples of the format: \\
  (date, correlation value, slope\_based value, linear\_regression value, 
graph\_based value, spike\_detection value, multiple\_arima value)
  \item list2 \textit{(list)} :  List of tuples of the format:  \\
  (date, correlation value, slope\_based value, linear\_regression value, 
graph\_based value, spike\_detection value, multiple\_arima value)

 \end{enumerate}

 \item Output \textit{(list)}: \\
 This function returns union of list1 and list2. Returned value is list of 
tuples of the form: \\
 (date, correlation value, slope\_based value, linear\_regression value, 
graph\_based value, spike\_detection value, multiple\_arima value)

 \end{itemize}
 
 
 
 
  \item mergeDates(li) \\
  
  
  This function merges overlapping time period. The list, which it takes as 
input, ``li'', is list of tuples, of the format, \\
  (startDate, endDate, Value). \\
  \\
  So if, we have overlapping period or two time periods are adjacent, for 
example if one tuple is (1-1-2015, 15-1-2015, 5) and other tuple is (15-1-2015, 
30-1-2015, 6), than this function will produce output as, 
(1-1-2015, 30-1-2015, 5).
  
  \begin{itemize}
 \item Input Parameters
 
 \begin{enumerate}
  \item li \textit{(list)} : List of tuples of the format: \\
  (startDate, endDate, Value) \\
  Note that here startDate and endDate, are of type \textit{datetime} and Value 
is of type \textit{float}.
 \end{enumerate}

 \item Output \textit{(list)}: \\
 This function returns list of tuples of the form: \\
 (startDate, endDate, Value)

 \end{itemize}
  
  
  
  
 \item resultOfOneMethod(array) \\
 
 
 This function just converts format of the result list. Usually, anomaly 
detection methods returns list of tuples of the format, \\ 
 (startDate, endDate, Value) \\
 this function will convert it to list of tuples of the format, \\
 (date, value) \\
 \\
 Basically, all the dates between startDate and endDate will be added to the 
result list.
 
 \begin{itemize}
 \item Input Parameters
 
 \begin{enumerate}
  \item array \textit{(list)} : List of tuples of the format: \\
  (startDate, endDate, Value) \\
  Note that here startDate and endDate, are of type \textit{datetime} and Value 
is of type \textit{float}.
 \end{enumerate}

 \item Output \textit{(list)}: \\
 This function returns list of tuples of the form: \\
 (date, Value)
  Note that here date is of type \textit{datetime} and Value is of type 
\textit{float}.
 \end{itemize}
 
 
 
 
 
 
 
 
 
 
 
 
 
 
 
 
 
 
 
\end{itemize}


