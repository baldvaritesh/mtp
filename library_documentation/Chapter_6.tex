\chapter{Utility}

Refer file ``library/Utility.py''.

\section{Introduction}
         
Here, we explain some related funtions which are used by stated anomaly 
detection techniques. Note that some of these functions can be used to process 
the results of the anomaly techniques.

\subsection{Functions used by Anomaly detection techniques}

\begin{itemize}
 \item \textbf{convertListToFloat(li)} \\
	Given list of elements, this function type casts all the elements to 
float type.
	
	\begin{itemize}
	  \item Input Parameters
	  
	  \begin{enumerate}
	    \item li \textit{(int)} : list of elements	    
	  \end{enumerate}

	  \item Output \textit{(list)}: \\
	  Returns list of elements after converting each element into float 

	  \end{itemize}
    
	
	
	
	
 \item \textbf{getColumnFromListOfTuples(lstTuples,i)} \\
 
    This function returns i'th element of all tuples as a list.
    
    \begin{itemize}
	  \item Input Parameters
	  
	  \begin{enumerate}
	    \item lstTuples \textit{(int)} : list of tuples. Tuples has no fixed 
format
	    \item i \textit{(int)} : index of which tuple element to return, 
index starting from zero
	  \end{enumerate}

	  \item Output \textit{(list)}: \\
	  Returns list of elements after fetching i'th element from each tuple.
	  
    \end{itemize}
 
 
 \item \textbf{findAverageTimeSeries(timeSeriesCollection)} \\
 
	It takes 2D list of element, where each element of timeSeriesCollection 
is one time series. It returns average of all time series. (first element of 
resltant time series will be average of first element of all time series) \\
	
	For example let,\\
	timeSeriesCollection: [ \newline
	    [1,2,3], \# Timeseries 1 \newline
	    [4,5,6], \# Timeseries 2 \newline
	    [7,8,9] \# Timeseries 3 \newline
	] \newline
	\\
	This function will return,\newline
	[4,5,6] \newline
	
	\begin{itemize}
	  \item Input Parameters
	  
	  \begin{enumerate}
	    \item timeSeriesCollection \textit{(list)} : 2D array of float 
elements.
	  \end{enumerate}

	  \item Output \textit{(list)}: \\
	  Returns list after taking average of all time series.
	  
	\end{itemize}
 
 
 
 \item \textbf{writeToCSV(lstData,fileName)} \\
 
 This writes the list of tuples into the file provided as input.
 \begin{itemize}
	  \item Input Parameters
	  
	  \begin{enumerate}
	    \item lstData \textit{(list)} : list of tuples that needs to be written in csv file.
	    \item fileName \textit{(string)} : Name of file in which the list of tuples needs to be written.
	    \end{enumerate}

	  \item Output \textit{(file)}: \\
	  Generates a csv file with data written in that.
	  
	\end{itemize}

 \item \textbf{concateLists(lstData)} \\
 
  This function converts the list of lists into a single list of tuples. 
  
  For example let,\\
	timeSeriesCollection: [ \newline
	    [1,2,3], \# Timeseries 1 \newline
	    [4,5,6], \# Timeseries 2 \newline
	    [7,8,9] \# Timeseries 3 \newline
	] \newline
	\\
	This function will return,\newline
	[ \newline
	    (1,4,7), \# Timeseries 1 \newline
	    (2,5,8), \# Timeseries 2 \newline
	    (3,6,9) \# Timeseries 3 \newline
	] \newline
  
 \begin{itemize}
	  \item Input Parameters
	  
	  \begin{enumerate}
	    \item lstData \textit{(list)} : list of different lists
	    \end{enumerate}

	  \item Output \textit{(file)}: \\
	  Return a single list of tuples.
	  
	\end{itemize}
	
 \item \textbf{cleanArray(array)} \\
 
	This function removes ``nan'' (Not a number values) from the list.
	
	\begin{itemize}
	  \item Input Parameters
	  
	  \begin{enumerate}
	    \item array \textit{(list)} : list of float type elements
elements.
	  \end{enumerate}

	  \item Output \textit{(list)}: \\
	  Returns list of float elements after removing ``nan'' elements.
	  
	\end{itemize}
 
 
 \item \textbf{MADThreshold(array)} \\
 
 This function is used to calculate threshold value using Median Absolute Deviation outlier detection method.
 
 \begin{itemize}
	  \item Input Parameters
	  
	  \begin{enumerate}
	    \item array \textit{(list)} : Array of integers, real numbers, etc
	    \end{enumerate}

	  \item Output \textit{(float)}: \\
	  threshold value computed using MAD Test.
	  
	\end{itemize}
 
 \item \textbf{smoothArray(array, alpha = 2.0/15.0)} \\
 This function smooths input array by exponential moving average technique.
 
 \begin{itemize}
	  \item Input Parameters
	  
	  \begin{enumerate}
	    \item array \textit{(list)} : Array of integers, real numbers, etc
	    \item alpha \textit{(float)} : smoothening factor of exponential average smoothing
	    \end{enumerate}

	  \item Output \textit{(list)}: \\
	  Array which is exponentially smoothed
	  
	\end{itemize}
 
 
 \item \textbf{csv2array(filePath) }\\
 
  This function reads csv file into list
 
 \begin{itemize}
	  \item Input Parameters
	  
	  \begin{enumerate}
	    \item filePath \textit{(string)} : Path of file to be read
	    \end{enumerate}

	  \item Output \textit{(list)}: \\
	  list of rows in the csv file
	  
	\end{itemize}
 
 \item \textbf{getColumn(array, column\_number)} \\
 
   This function fetches a particular column from a list of tuples.
 
 \begin{itemize}
	  \item Input Parameters
	  
	  \begin{enumerate}
	    \item array \textit{(list)} : List of tuples
	    \item column\_number \textit{(int)} : The index of the column number to be fetched from list of tuples
	    \end{enumerate}

	  \item Output \textit{(list)}: \\
	  list of elements corresponding to the column
	  
	\end{itemize}
	
 \item \textbf{formatCSV2Array(z)} \\
 
    This function returns the same list with changed datatypes of elements within it.
 
 \begin{itemize}
	  \item Input Parameters
	  
	  \begin{enumerate}
	    \item z \textit{(list)} : List of tuples
	    \end{enumerate}

	  \item Output \textit{(list)}: \\
	  return same array changing data type of second column to float
	  
	\end{itemize}
 

\end{itemize}


\subsection{Functions used to process results}

\begin{itemize}
   
   
   \item \textbf{intersection(numOfResults, list1, resultOf1, list2, resultOf2, list3 = 
[], resultOf3="linear\_regression", list4=[], resultOf4="graph\_based", 
list5=[], resultOf5="spike\_detection" , list6=[], resultOf6="multiple\_arima") }
\\
  This function is used to take intersection of results of multiple methods (2 
or more), which are passed as list here. This function requires minimum of 5 
arguments. 
   
   \begin{itemize}
 \item Input Parameters
 
 \begin{enumerate}
  \item numOfResults \textit{(int)} : number of lists you are passing, minimum 2
  \item $list_i$ \textit{(list)} : list represnting result of the i'th 
algorithm. Note that this is list of tuples of the format, \\
  (startDate, endDate, value)
  \item $resultOf_i$ \textit{(string)}: this variable states, $list_i$ is of 
which algorithm, it can be from following: \\
            (slope\_based, linear\_regression, graph\_based, spike\_detection, 
multiple\_arima)

 \end{enumerate}

 \item Output \textit{(list)}: \\
 This function returns intersection of all lists. Returned value is list of 
tuples of the form: \\
 (date, correlation value, slope\_based value, linear\_regression value, 
graph\_based value, spike\_detection value, multiple\_arima value)

 \end{itemize}
   
   
   
   
   \item \textbf{intersectionOfFinalResults(list1, list2)} \\
  
  This function takes intersection of 2 lists (where each list is list of 
tuple) and returns result. Note that these input lists are generated as a output 
of ``intersection'' method.
 \begin{itemize}
 \item Input Parameters
 
 \begin{enumerate}
  \item list1 \textit{(list)} : List of tuples of the format: \\
  (date, correlation value, slope\_based value, linear\_regression value, 
graph\_based value, spike\_detection value, multiple\_arima value)
  \item list2 \textit{(list)} :  List of tuples of the format:  \\
  (date, correlation value, slope\_based value, linear\_regression value, 
graph\_based value, spike\_detection value, multiple\_arima value)

 \end{enumerate}

 \item Output \textit{(list)}: \\
 This function returns intersection of list1 and list2. Returned value is list 
of tuples of the form: \\
 (date, correlation value, slope\_based value, linear\_regression value, 
graph\_based value, spike\_detection value, multiple\_arima value)

 \end{itemize}
 
 
 
 \item \textbf{unionOfFinalResults(list1, list2) }\\
 This function takes union of 2 lists (where each list is list of tuple) and 
returns result. Note that these input lists are generated as a output of 
``intersection'' method.
 \begin{itemize}
 \item Input Parameters
 
 \begin{enumerate}
  \item list1 \textit{(list)} : List of tuples of the format: \\
  (date, correlation value, slope\_based value, linear\_regression value, 
graph\_based value, spike\_detection value, multiple\_arima value)
  \item list2 \textit{(list)} :  List of tuples of the format:  \\
  (date, correlation value, slope\_based value, linear\_regression value, 
graph\_based value, spike\_detection value, multiple\_arima value)

 \end{enumerate}

 \item Output \textit{(list)}: \\
 This function returns union of list1 and list2. Returned value is list of 
tuples of the form: \\
 (date, correlation value, slope\_based value, linear\_regression value, 
graph\_based value, spike\_detection value, multiple\_arima value)

 \end{itemize}
 
 
 
 
  \item \textbf{mergeDates(li)} \\
  
  
  This function merges overlapping time period. The list, which it takes as 
input, ``li'', is list of tuples, of the format, \\
  (startDate, endDate, Value). \\
  \\
  So if, we have overlapping period or two time periods are adjacent, for 
example if one tuple is (1-1-2015, 15-1-2015, 5) and other tuple is (15-1-2015, 
30-1-2015, 6), than this function will produce output as, 
(1-1-2015, 30-1-2015, 5).
  
  \begin{itemize}
 \item Input Parameters
 
 \begin{enumerate}
  \item li \textit{(list)} : List of tuples of the format: \\
  (startDate, endDate, Value) \\
  Note that here startDate and endDate, are of type \textit{datetime} and Value 
is of type \textit{float}.
 \end{enumerate}

 \item Output \textit{(list)}: \\
 This function returns list of tuples of the form: \\
 (startDate, endDate, Value)

 \end{itemize}
  
  
  
  
 \item \textbf{resultOfOneMethod(array)} \\
 
 
 This function just converts format of the result list. Usually, anomaly 
detection methods returns list of tuples of the format, \\ 
 (startDate, endDate, Value) \\
 this function will convert it to list of tuples of the format, \\
 (date, value) \\
 \\
 Basically, all the dates between startDate and endDate will be added to the 
result list.
 
 \begin{itemize}
 \item Input Parameters
 
 \begin{enumerate}
  \item array \textit{(list)} : List of tuples of the format: \\
  (startDate, endDate, Value) \\
  Note that here startDate and endDate, are of type \textit{datetime} and Value 
is of type \textit{float}.
 \end{enumerate}

 \item Output \textit{(list)}: \\
 This function returns list of tuples of the form: \\
 (date, Value)
  Note that here date is of type \textit{datetime} and Value is of type 
\textit{float}.
 \end{itemize}
 
 
 
 \item \textbf{plotGraphForHypothesis(original, average, list1, list2, total\_news\_articles)} \\
  
 Given the arguments of 2 series, anomalies matched, anomalies reported and articles which are not matched, this function plots graph of that. Note that here here both the series should have same units.
 
 \begin{itemize}
 \item Input Parameters
 
 \begin{enumerate}
  \item original: Series 1
  \item average: Series 2
  \item list1: Predicted By system : list of tuples ... (date, ... , ...)
  \item list2: Actual results : List of Dates
  \item total\_news\_articles: All articles present in the database for this center/entity
 \end{enumerate}

 \item Output \textit{(list)}: \\
 Plots Graph.
 \end{itemize}
 
 
 \item \textbf{fetchNewsForCenter(resultsOfSystem, centerNumber, intervalToConsider=5)} \\
  
 Given the list of anomalies and name of center for which anomalies are reported, this function returns the all news articles present for this centre and all the anomalies which are matched with anomalies.\\
 
 \begin{itemize}
 \item Input Parameters
 
 \begin{enumerate}
  \item resultsOfSystem: List of tuples/list of the format (date, ... ,...)
  \item centerNumber: Center Number (Refer placeMapping function for more details about centre number)
  \item intervalToConsider: News will be fetched from news articles for a particular date(given by system as anomaly) in interval [date-intervalToConsider, date+intervalToConsider]
 \end{enumerate}

 \item Output \textit{(list)}: \\
 returns result of the following form....
A tuple:

	(resultList, allArticlesQueryResult)
	
Where:\\
\\
* resultList : This is of tuples with following fields:\\
				(System\_anomaly\_date, news\_article\_date, \\news\_source, source\_url, difference\_between\_system\_date\_and\_news\_article\_date \\,reason, comment, days\_compared\_in\_article)\\
				1. System\_anomaly\_date: date reported by our system which is reported as anomolous date\\
				2. news\_article\_date: date of news article corresponding to above System\_anomaly\_date\\
				3. news\_source: source of news (which media?)\\
				4. source\_url: link of the news article\\
				5. difference\_between\_system\_date\_and\_news\_article\_date: difference between dates of (news\_article\_date - System\_anomaly\_date)\\
				6. reason: what reason is stated by article\\
				7. comment: any comment on article if present\\
				8. days\_compared\_in\_article: Article has compared data between how many days to state that it is anomaly?\\
				\\
* allArticlesQueryResult: List of dates of all news articles for this center which is present in the database
 \end{itemize}
 


 \item \textbf{plotGraphForHypothesisArrival(original, average, list1, list2, total\_news\_articles)} \\
  
 Given the arguments of 2 series, anomalies matched, anomalies reported and articles which are not matched, this function plots graph of that. Note that here here both the series have different units like one series is of price and other is of arrival.
 
 \begin{itemize}
 \item Input Parameters
 
 \begin{enumerate}
  \item original: Series 1
  \item average: Series 2
  \item list1: Predicted By system : list of tuples ... (date, ... , ...)
  \item list2: Actual results : List of Dates
  \item total\_news\_articles: All articles present in the database for this center/entity
 \end{enumerate}

 \item Output \textit{(list)}: \\
 Plots Graph.
 \end{itemize} 
 
 
 
 \item \textbf{placeMapping(i)} \\
  
 Depending upon the value of parameters passed, it returns centre name as follows:\\
 0: Mumbai\\
 1: Delhi\\
 2: Ahmedabad\\
 3: Banglore\\
 4: Patna\\
 5: India\\
 
 
 
 \item \textbf{union(numOfLists, *lists)} \\
  
 Takes union of dates, where list of dates are passed as list of tuples where first element of tuple is date.
 
 \begin{itemize}
 \item Input Parameters
 
 \begin{enumerate}
  	\item numOfLists: It states number of lists, whose union is to be found
	\item lists: each element of lists is a list, whose union is to be found, each of this list should be list of tuples, whose first element should be date
 \end{enumerate}

 \item Output \textit{(list)}: \\
 returns list of dates after union
 \end{itemize}
 

 \item \textbf{intersect(list1,list2)} \\
  
 Takes intersection of elements present in lists list1 and list2.
 
 \begin{itemize}
 \item Input Parameters
 
 \begin{enumerate}
  	\item list1 : list of dates in tuple format (date,)
	\item list2 : list of dates in tuple format (date,)
 \end{enumerate}

 \item Output \textit{(list)}: \\
 returns intersection of list1 and list2
 \end{itemize} 
 
 
 
 
 
 
\end{itemize}


