\chapter{Graph Based Anomaly Detection Technique}

\section{Introduction}

This technique was introduced by \cite{nasa}. We have used is R implementation given by 
authors of this book \cite{nasarbook}. So, here by using python script, we will be just 
calling R script with appropriate arguments and will be using result provided 
by that script.

Graph based anomaly detection technique considers each day as a  node of a 
graph. Similar nodes are connected to each other by some weight. Similarity of 
nodes are calculated by making use of the values of that node i.e. value(s) of 
timeseries on that date. Based on this similarity, edge weights are also 
assigned. Than random walk algorithm is applied on this graph structure and 
connectivity value of each node is calculated. Graph nodes having the least 
connectivity values are reported as anomaly.

Note that previous techniques, like Window 
Correlation, Slope Based and Linear Regression techniques, can take only 2 time 
series as input. They also don't consider historical values, trend or 
seasonality. It just makes prediction on the given present data. Where as, this 
Graph based anomaly detection technique, can take multiple time series as input 
and also considers trends, seasonality as well, as explained in research paper 
\cite{nasa}.

So, here, we take multiple time series as input. Out of them, one will be 
dependent on rest of the others. We will call R script, it will print result in 
one csv file. We read that CSV file and return result. Note that here we do not 
have threshold value. We just give number of points with the least connectivity 
value and function returns them. If in future, one wants to add threshold value 
on connectivity than function can be modified according to that as well.

\section{Related Functions}

\subsection{graphBasedAnomalyCall(dependentVar, numberOfVals, 
timeSeriesFileNames)}

This function calls the R Script ``graphBasedAnomaly.R''. This function takes 
multiple time series as input, which are stored in files, whose name are stored 
in ``timeSeriesFileNames'' list. This time-series files are generated by us 
only. Out of these time series, one will be for dependent variable and others 
will be corresponding to independent variable. So variable, ``dependentVar'' 
represents which time series/variable is dependent.

this function executes R script and writes output to the file named 
``GraphBasedAnomalyOp.csv''.

\begin{itemize}
 \item Input Parameters
 
 \begin{enumerate}
  \item dependentVar \textit{(int)} : Index of the dependent variable, where 
dependentVar = function of independantVars
  \item numberOfVals \textit{(int)} : Each CSV contains how many values? That 
is each time series has how many values?
  \item timeSeriesFileNames \textit{(list)} : Name of the files to which series 
is stored. File should contain only series values.

 \end{enumerate}

 \item Output: This function does not generate any output. R Script will write 
output to CSV file as stated before.

 
\end{itemize}


\subsection{generateCSVsForGraphBasedAnomaly(lists, dateIndex, seriesIndex)}

In python code, we have time-series as a list. This list is list of tuples, in 
which first value of tuple is date and than we have more than one values in the 
same tuple, representing different time-series. For example, if we have 
test-case as onion, than for one city we have 3 time series along with date, 
which is rpresented as list of tuples of the the form (date, arrival, wholesale 
price, retail price). But, for R script, we need just time series values. So 
this function will take series of time series in variable ``lists``, where 
lists[i] will represent one timeseries or multiple time series for one object 
(like explained previously we can have multiple time series for one city).

dateIndex will say which tuple number for the list lists[i] represents date and 
seriesIndex represents, if lists[i] represents multiple series than which one 
to take out of them. This can be explained by example as follows:

Let's say, we have lists as folows: \newline
\newline
[ \newline
\hfill  [(1-1-2010, x1, y1, z1), (2-1-2010, x2, y2, z2), ... ], \newline
\hfill  [(1-1-2010, x1, y1, z1), (2-1-2010, x2, y2, z2), ... ], \newline
\hfill  [...], ...  \newline
]; \newline
 \newline
So, here we have time-series corresponding to two entities, which can be 
accessed via lists[0] and lists[1]. Now, lists[0] gives us 3 time-series for 
one entity. But let's say, here we need only one corresponding to ''y`` time 
series. So, give dateIndex as 0 here and seriesIndex as 2. So, this function 
will create 2 CSVs, one for each entity. Each CSV will have values [y1, y2, y3, 
...]. One line will contain one value in file.

Note that it is not necessary to have multiple time-series for on entity. We 
can have just simple structure as follows: \newline
\newline
[ \newline
\hspace{1cm}  [(1-1-2010, x1), (2-1-2010, x2), ... ], \newline
\hspace{1cm}  [(1-1-2010, y1), (2-1-2010, y2), ... ], \newline
\hspace{1cm}  [...], ...  \newline
]; \newline
 \newline
So here, we have two time-series as x and y, and we can than give dateIndex as 
0 here and seriesIndex as 1. This will create two CSVs, one for ''x`` and other 
for ''y``.

After creating these CSVs, this function returns names of the file created.


\begin{itemize}
 \item Input Parameters
 
 \begin{enumerate}
  \item lists \textit{(list)} : List of time-series, where lists[i] = list of 
tuple of the form (date, val1 [, val2, val3, ...]) \\
    where date is in form of string and values in square brackets are optional.
  \item dateIndex \textit{(int)} : column number of date in list of tuple 
(starting with 0)
  \item seriesIndex \textit{(int)} : column number of series in list of tuple 
(starting with 0)

 \end{enumerate}

 \item Output \textit{(Tuple)}: \\
  returns tuple of the form, (dates,fileNames), \\
  
  Where, \\
  fileNames: Generated multiple CSVs, corresponding to each series for the 
input of R script. Returns name of these files. \\
  dates: Separated date from the series, so that later we can combine result of 
the R script (anomalies) with dates.

 
\end{itemize}


\subsection{getAnomalies(dates,resultFile, numOfPtsReqd)}

This function does the work of combining result of R script with the date. 
Result generated by R will be in some file, which is passed here as resultFile 
parameter. This will have indices for each day. So using this we append dates 
to it. So now, we have connectivity value for each date. This function sorts 
them according to connectivity value and returns the number of points required 
stated by parameter numOfPtsReqd, which as low connectivity value. 


\begin{itemize}
 \item Input Parameters
 
 \begin{enumerate}
  \item dates \textit{(list)} : List of dates, returned by ''getAnomalies`` 
function.
  \item resultFile \textit{(string)} : Path of file to which output of R script 
is written
  \item numOfPtsReqd \textit{(int)} : Number of anomalous points required

 \end{enumerate}

 \item Output \textit{(list)}: \\
  reurns list of tuples of the form: \\
  (start\_date, end\_date, connectivity\_value) \\
  Note that, here start\_date will be same as end\_date, as this function 
returns results day-wise.
  
\end{itemize}

\subsection{graphBasedAnomalyMain(lists, dependentVar, numOfPtsReqd, 
dateIndex=0, seriesIndex=1)}

This is the main function of this method. This function makes call to other 
functions, uses the output of one, as a input to other, combines all functions 
and returns generated output.

\begin{itemize}
 \item Input Parameters
 
 \begin{enumerate}
  \item lists \textit{(list)} : List of time-series, where lists[i] = list of 
tuple of the form (date, val1 [, val2, val3, ...]) \\
    where date is in form of string and values in square brackets are optional. 
   
  \item dependentVar \textit{(int)} : Index of the dependent variable, where 
dependentVar = function of independantVars
  \item numOfPtsReqd \textit{(int)} : Number of anomalous points required
  \item dateIndex \textit{(int)} : column number of date in list of tuple 
(starting with 0)
  \item seriesIndex \textit{(int)} : column number of series in list of tuple 
(starting with 0)
  
 \end{enumerate}

\end{itemize}

\section{Description}

Putting all together, here is the summary:\\
\\
''graphBasedAnomalyMain`` is the main function. First, it calls 
''generateCSVsForGraphBasedAnomaly``, which will generate files for each 
series, which will be used as input for R script. It also generates list of 
dates. Now, this list of file is passed to function, ''graphBasedAnomalyCall``, 
which will execute R script and will generate output in predefined file. This 
file name, along with dates and number of anomaly points required is passed to 
function  ''getAnomalies``, which will return output in required format.