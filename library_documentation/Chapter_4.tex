\chapter{Graph Based Anomaly Detection Technique}

\section{Introduction}

This technique was introduced by [1]. We have used is R implementation given by 
authors of this book [2]. So, here by using python script, we will be just 
calling R script with appropriate arguments and will be using result provided 
by that script.

Graph based anomaly detection technique considers each day as a  node of a 
graph. Similar nodes are connected to each other by some weight. Similarity of 
nodes are calculated by making use of the values of that node i.e. value(s) of 
timeseries on that date. Based on this similarity, edge weights are also 
assigned. Than random walk algorithm is applied on this graph structure and 
connectivity value of each node is calculated. Graph nodes having the least 
connectivity values are reported as anomaly.

Note that previous techniques, like Window 
Correlation, Slope Based and Linear Regression techniques, can take only 2 time 
series as input. They also don't consider historical values, trend or 
seasonality. It just makes prediction on the given present data. Where as, this 
Graph based anomaly detection technique, can take multiple time series as input 
and also considers trends, seasonality as well, as explained in research paper 
[1].

So, here, we take multiple time series as input. Out of them, one will be 
dependent on rest of the others. We will call R script, it will print result in 
one csv file. We read that CSV file and return result. Note that here we do not 
have threshold value. We just give number of points with the least connectivity 
value and function returns them. If in future, one wants to add threshold value 
on connectivity than function can be modified according to that as well.

\section{Related Functions}

\subsection{graphBasedAnomalyCall(dependentVar, numberOfVals, 
timeSeriesFileNames)}

This function calls the R Script ``graphBasedAnomaly.R''. This function takes 
multiple time series as input, which are stored in files, whose name are stored 
in ``timeSeriesFileNames'' list. This time-series files are generated by us 
only. Out of these time series, one will be for dependent variable and others 
will be corresponding to independent variable. So variable, ``dependentVar'' 
represents which time series/variable is dependent.

this function executes R script and writes output to the file named 
``GraphBasedAnomalyOp.csv''.

\begin{itemize}
 \item Input Parameters
 
 \begin{enumerate}
  \item dependentVar \textit{(int)} : Index of the dependent variable, where 
dependentVar = function of independantVars
  \item numberOfVals \textit{(int)} : Each CSV contains how many values? That is 
each time series has how many values?
  \item timeSeriesFileNames \textit{(list)} : Name of the files to which series 
is stored. File should contain only series values.

 \end{enumerate}

 \item Output: This function does not generate any output. R Script will write 
output to CSV file as stated before.

 
\end{itemize}


\subsection{generateCSVsForGraphBasedAnomaly(lists, dateIndex, seriesIndex)}

In python code, we have time-series as a list. This list is list of tuples, in 
which first value of tuple is date and than we have more than one values in the 
same tuple, representing different time-series. For example, if we have 
test-case as onion, than for one city we have 3 time series along with date, 
which is rpresented as list of tuples of the the form (date, arrival, wholesale 
price, retail price). But, for R script, we need just time series 

\section{Description}

Putting all together, here is the summary:\\
\\
