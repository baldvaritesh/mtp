\documentclass[a4paper,10pt]{report}
%\documentclass[a4paper,10pt]{scrartcl}
\usepackage{array}
\usepackage{booktabs}
\usepackage{blindtext}
\usepackage[utf8]{inputenc}
\usepackage{commath}
\usepackage[nottoc]{tocbibind}
\usepackage{url}

\title{Anomaly Detection Library}
\author{Kapil Thakkar and Reshma Kumari}
\date{27th May, 2016}

\pdfinfo{%
  /Title    (Anomaly Detection Library)
  /Author   (Kapil Thakkar and Reshma Kumari)
  /Creator  (Kapil Thakkar and Reshma Kumari)
  /Producer (Kapil Thakkar and Reshma Kumari)
  /Subject  (Major Project)
  /Keywords (Anomaly Detection)
}

\begin{document}
\maketitle

\begin{abstract}

This document explains the methods and corresponding function written for Anomaly detection Library to detect anomaly in multiple time series. All corresonding files can be found in ``library'' folder. We have performed analysis on Onion data considering its Retail Price, Wholesale Price and Arrival data. File used to perform analysis and studying various results is ``library/fullAnalysis.py''. To study local and national anomalies file used is ``library/fullAnalysisUpdated.py''. ``fullAnalysis.py'' file contains required comments to understand how analysis is performed.

\end{abstract}


\tableofcontents{}

\chapter{Window Based Correlation}

\section{Introduction}

This technique is basically applied on two time-series. Let's say we have two 
time series as series1 and series2. So, in this method, we first find 
correlation at various lags between these two time series. User can specify 
minimum and maximum lag to consider. So, for all those values, we find 
correlation values. 

After finding correlation values at all lags, we consider that lag at which 
correlation value is higher among all previously calulated correlation values 
at all lags. Let's say that lag be ``x''. So, depending upon that ``x'', we 
shift series1 or series2. If ``x'' is positive, we move series2 by ``x'' units 
and if it is negative than we shift series1 by \abs{x} units.

Now, we are ready to apply window correlation. Take window value,``w'' as 
input. First window will be from 1st element to w'th elementof both the time 
series after aligning by lag ``x''. Find correlation for this window between 
two time-series and save it in an array. Now, slide window by ``w'' elements 
and 
calculate correlation value again and so on. Now, we have correlation values at 
multiple windows.

Now, let's say both the series should have been positively correlated. So, what 
we do is, we choose threshold by MAD test if not provided to us, and find all 
correlation values which are below that threshold and report all those windows 
as anomaly.

\section{Related Functions}

\subsection{correlation(arr1, arr2, maxlag, pos, neg)}

This function calculates correlation between arr1 and arr2 at all possible lags 
between -maxlag to +maxlag, as specified by pos and neg parameters.

\begin{itemize}
 \item Input Parameters
 
 \begin{enumerate}
  \item arr1 \textit{(list)} : Input series 1 as a list of float values
  \item arr2 \textit{(list)} : Input series 2 as a list of float values
  \item maxlag \textit{(int)} : maximum (maxlag) and minimum (-maxlag) lag to 
consider while calculating correlation betweem arr1 and arr2
  \item pos \textit{(int, 1 or 0)} : To consider positive lag or not, i.e. 1 to 
maxlag
  \item neg \textit{(int, 1 or 0)} : To consider negative lag or not, i.e. 
-maxlag to -1
 \end{enumerate}

 \item Output \textit{(list)} : \\
  Returns list of tuples of the form \\
  \center{(lag, correlation value at this lag)}
 
\end{itemize}

\subsection{getMaxCorr(arar1,positive\_correlation)}

This function takes list of tuples of the form (lag, correlation value at this 
lag) as  input. Returns lag value at which correlation value is maximum if 
positive\_correlation is True, and returns lag at which correlation value is 
minimum if positive\_correlation is False. \\
\\
Basically, if both the series are positively correlated than we will be 
interested in maximum positive correlation or if both series are negatively 
correlated than we will be interested in minimum negative correlation, which is 
specified by positive\_correlation parameter.

\begin{itemize}
 \item Input Parameters
 
 \begin{enumerate}
  \item arr1 \textit{(list)} : list of tuples of the form \\ (lag, correlation 
value at this lag) \\ i.e. correlation values at various lags
  \item positive\_correlation \textit{(boolean, ``True'' or ``False'')} : 
      \begin{itemize}
       \item True: If value of this parameter is True than it will return lag 
at which correlation value if maximum (positive)
       \item False: If value of this parameter is False than it will return lag 
at which correlation value if minimum (negative)
      \end{itemize}

 \end{enumerate}

 \item Output \textit{(Tuple)} : \\
  returns single tuple of the form (lag,correlation value at this lag), i.e. 
lag at which optimum correlation value is found along with correlation value.
 
\end{itemize}


\subsection{correlationAtLag(series1, series2, lag, window\_size)}

This function fisrt aligns two series by given lag. If lag is positive than it 
shifts start of series2 else start of series1. After aligning both the series 
according to lag, this function calculates correlation between both series at 
all windows. 

window\_size states size of the window. So, we will start with first window 
taking first window\_size elements from each series and will calculate 
correlation. We will save this correlation value in list and will slide to next 
window. Next window will start after window\_size elements. In such a way, we 
calculate, correlation at all windows and return the list of correlation values.


\begin{itemize}
 \item Input Parameters
 
 \begin{enumerate}
  \item series1 \textit{(list)} : Input series 1 as a list of float values
  \item series2 \textit{(list)} : Input series 2 as a list of float values
  \item lag \textit{(int)} : lag at which series needs to be adjusted as 
explained above
  \item window\_size \textit{(int)} : window size to be considered
 \end{enumerate}

 \item Output \textit{(list)} : \\
  Returns list of correlation values (of float type) for all windows calculated 
at given lag \\
 
\end{itemize}


\subsection{WindowCorrelationWithConstantLag(arr1, arr2, window\_size,maxlag, 
positive\_correlation, pos, neg)}

This is sort of driver function, which will call above 3 functions. This 
function will first get lag at which series needs to be adjusted. Than using 
this lag, it will calculate correlation values at all windows and will return 
it.

\begin{itemize}
 \item Input Parameters
 
 \begin{enumerate}
  \item arr1 \textit{(list)} : Input series 1 as a list of float values
  \item arr2 \textit{(list)} : Input series 2 as a list of float values
  \item window\_size \textit{(int)} : window size to be considered while 
calculating window correlation
  \item maxlag \textit{(int)} : maximum (maxlag) and minimum (-maxlag) lag to 
consider while calculating correlation betweem arr1 and arr2, to align both the 
series
  \item positive\_correlation \textit{(boolean, ``True'' or ``False'')} : 
      \begin{itemize}
       \item True: This suggest that both the series are positively correlated
       \item False: This suggest that both the series are negatively correlated
      \end{itemize}
      
  \item pos \textit{(int, 1 or 0)} : If value of this parameter is True than we 
will consider positive values for lag, i.e. 1 to +maxlag to align both the 
series initially
  \item neg \textit{(int, 1 or 0)} : If value of this parameter is True than we 
will consider negative values for lag, i.e. -maxlag to -1 to align both the 
series initially
  
 \end{enumerate}

 \item Output \textit{(list)} : \\
  List of tuples of the form (lag,array)
Where lag is lag value for which whole series is shifted and then at that lag, 
we have calculated correlation for all window. Correlation value for all 
windows is stored in array.
 
\end{itemize}

\subsection{anomaliesFromWindowCorrelationWithConstantlag(arr1, arr2, 
window\_size=15,maxlag=15, positive\_correlation=True, pos=1, neg=1, 
default\_threshold = True, threshold = 0):}


This is main function of this method. This is driver of whole method. Using 
previously stated methods, it will first gather correlation values at different 
windows. Than depending upon which type of threshold is to be used, it will 
filter out anomalies. If default threshold is to be used, than it will be 
caulated using MAD test on the correlation values at each window, else 
threshold provided by user will be used. 

Correlation values not satisying threshold will be reported along with the date 
range of that window.


\begin{itemize}
 \item Input Parameters
 
 \begin{enumerate}
  \item arr1 \textit{(list)} : Input series 1 as a list of tuples of the from 
(date,value)
  \item arr2 \textit{(list)} : Input series 2 as a list of tuples of the from 
(date,value)
  \item window\_size \textit{(int)} : window size to be considered while 
calculating window correlation
  \item maxlag \textit{(int)} : maximum (maxlag) and minimum (-maxlag) lag to 
consider while calculating correlation betweem arr1 and arr2, to align both the 
series
  \item positive\_correlation \textit{(boolean, ``True'' or ``False'')} : 
      \begin{itemize}
       \item True: This suggest that both the series are positively correlated
       \item False: This suggest that both the series are negatively correlated
      \end{itemize}
      
  \item pos \textit{(int, 1 or 0)} : If value of this parameter is True than we 
will consider positive values for lag, i.e. 1 to +maxlag to align both the 
series initially
  \item neg \textit{(int, 1 or 0)} : If value of this parameter is True than we 
will consider negative values for lag, i.e. -maxlag to -1 to align both the 
series initially
  \item default\_threshold \textit{(boolean, ``True'' or ``False'')} : whether 
to use default threshold or not. If True, default threshold will be used using 
MAD test on calculated correlation values for all windows.
  \item threshold \textit{(float)} : if default\_threshold is False, than this 
user provided threshold will be used.
 \end{enumerate}

 \item Output \textit{(list)} : \\
  This function filter out anomalies and returns them. This function returns
  List of tuples of the form \\ (start\_date,end\_date,correlation\_value), \\
  where (start\_date, end\_date) specifies range of the window and 
correlation\_value if value of correlation of that window
  
\end{itemize}

\section{Description}


Putting all together, here is the summary:\\
\\
Function ''WindowCorrelationWithConstantLag``, first makes use of 
''correlation`` function, to calculate correlation values at all lags to find 
out at which lag it needs to be align. Result of ''correlation`` function is 
passed to ''getMaxCorr`` function. Which will return lag at which optimum value 
of correlation is present. This output will be used by ''correlationAtLag`` 
function, to caluclate correlation at all windows after aligning both series by 
input lag. So, in this way ''WindowCorrelationWithConstantLag`` combines these 
three functions and returns correlation value at each window.
\\
\\
Function ``anomaliesFromWindowCorrelationWithConstantlag`` is the main driver. 
This function calls ''WindowCorrelationWithConstantLag`` and gets the 
correlation values at all windows and filters out anomalies (either using 
threshold calculated by MAD test or by user provided threshold) and returns them 
in the format of (start\_date,end\_date,correlation\_value),
  where (start\_date, end\_date) specifies range of the window and 
correlation\_value if value of correlation of that window. 
\chapter{Slope Based Detection}

Refer file ``library/slopeBasedDetection.py''.

\section{Introduction}

The method works on two time-series. It finds the ratio of steepness 
at two different points in the time-series. Let's say we have two 
time series as series1 and series2. So in this method, we first find 
the rate of change in the time-series values for both time-series followed 
by taking ratio of these rate of change. i.e  suppose we have two points on 
time-series 1 as $y_{11}$ and $y_{12}$ and on time-series 2 as $y_{21}$ and $y_{22}$. 
Rate of change between these points is calculated as following
 $$S_1=\frac{y_{12}-y_{11}}{y_{11}}$$
 ,
  $$S_2=\frac{y_{22}-y_{21}}{y_{21}}$$
 Ratio of steepness ( \textbf{rs} ) is calculated as
 $$rs = \frac{S_1}{S_2}$$

The \textbf{rs} is calulated between first and last point of every window of
size ``w'' provided as input. 

Now, we have rate of change in the steepness(\textbf{rs}) for every window. 
The outliers are detected by the threshold value provided by user or if not, 
then threshold is computed using MAD test on the all \textbf{rs} calculated above.

If the two time-series are expected to move in tandem, then all the points 
with \textbf{rs} greater than threshold are reported whereas if the two time 
series should not move in tandem then all the points with \textbf{rs} less than 
threshold are reported. 

\section{Related Functions}

\subsection{slopeBasedDetection(series1,smoothed1,series2,smoothed2, \\ next\_val\_to\_consider, default\_threshold, threshold, what\_to\_consider)}

This function smoothes the provided time series data using exponential moving average(if needed) and 
calculates \textbf{rs} for first and last point of every window of size next{\_}val{\_}to{\_}consider.

% Since, we calculate rs at regular interval, to avoid sudden spike or steepness in result, we prefer smoothed time-series.

\begin{itemize}
 \item Input Parameters
 
 \begin{enumerate}
  \item series1 \textit{(list)} : Input series 1 as a list of float values
  \item smoothed1 \textit{(boolean)} : Whether series1 is smoothed or not? If not (value of this parameter is \textit{False} ) smoothing will be done
  \item series2 \textit{(list)} : Input series 2 as a list of float values
  \item smoothed2 \textit{(boolean)} : Whether series2 is smoothed or not? If not (value of this parameter is \textit{False} ) smoothing will be done
  \item next{\_}val{\_}to{\_}consider \textit{(int)} : indicates the size of window or next point in time-series to calculate slope of steepness. Default is 7 days.
  \item default{\_}threshold \textit{(int)} : Whether to consider default threshold or not. If \textit{True}, the threshold is calculated using MAD Test.
  \item threshold \textit{(int)} : Threshold value to consider if default\_threshold is set to \textit{False}
  \item what{\_}to{\_}consider \textit{(int)} : Can be either 1,0 or -1. If the series are supposed to move in tandem, 1 is set otherwise -1 is set. In case we don't know the correlation between two, 0 is set.
  \end{enumerate}

 \item Output \textit{(list)} : \\
  Returns list of tuples of the form \\
  \center{(first,second,slope\_value)}\\
  \begin{flushleft}
 where first and second are the array index of passed series for which the \textbf{rs} is calculated.
 \end{flushleft}
\end{itemize}

\subsection{anomalyDatesSlopeBaseddetetion(slopeBasedResult,any\_series)}

This function basically takes result of ”slopeBasedDetection” as input along with
any series which is list of tuples of the form (date, value), and gives date to each
anomaly.\\

The result returned by ”slopeBasedDetection” function just provides index of
data point, which is reported as anomaly. But we have time series, so we need
to provide date, instead of index of data point. So, this function basically,
attaches each anomaly with its date and returns it.\\

\begin{itemize}
 \item Input Parameters
 
 \begin{enumerate}
  \item slopeBasedResult \textit{(list)} : This is list of anomalies reported by ”slopeBasedDetection” function.
  \item any\_series \textit{(list)} : Any list/series of tuples in the format (Date,Value), date will be used from this series to find date against each anomaly.
 \end{enumerate}

 \item Output \textit{(list)} : \\
 	Returns list of tuples of the following form: \\ 
 	(start\_date,end\_date,slope\_value)

\end{itemize}


\subsection{slopeBased(series1,smoothed1,series2,smoothed2,next\_val\_to\_consider, default\_threshold, threshold, what\_to\_consider)}

This is main function of this anomaly detection technique. This function first calls "slopeBasedDetection" function, gets list of anomalies. 
After that, it calls "anomalyDatesSlopeBaseddetetion" function to attach date with each anomaly and than returns result.

\begin{itemize}
 \item Input Parameters
 
 \begin{enumerate}
  \item series1 \textit{(list)} : Input series 1 as a list of tuples of the forms (date, value)
  \item smoothed1 \textit{(boolean)} : Whether series1 is smoothed or not? If not (value of this parameter is \textit{False} ) smoothing will be done
  \item series2 \textit{(list)} : Input series 2 as a list of tuples of the forms (date, value)
  \item smoothed2 \textit{(boolean)} : Whether series2 is smoothed or not? If not (value of this parameter is \textit{False} ) smoothing will be done
  \item next{\_}val{\_}to{\_}consider \textit{(int)} : indicates the size of window or next point in time-series to calculate slope of steepness. Default is 7 days.
  \item default{\_}threshold \textit{(int)} : Whether to consider default threshold or not. If \textit{True}, the threshold is calculated using MAD Test.
  \item threshold \textit{(int)} : Threshold value to consider if default\_threshold is set to \textit{False}
  \item what{\_}to{\_}consider \textit{(int)} : Can be either 1,0 or -1. If the series are supposed to move in tandem, 1 is set otherwise -1 is set. In case we don't know the correlation between two, 0 is set.
  \end{enumerate}

 \item Output \textit{(list)} : \\
 	Returns list of tuples of the following form: \\ 
 	(start\_date,end\_date,slope\_value)\\
 	where, start\_date and end\_date are points on which the steep was computed along with the slope\_value which was spotted as outlier.
 
\end{itemize}

\section{Description}

slopeBased function is called to find anomalies based on the rate of change in value. It calls slopeBasedDetection method to compute the slopes between points in windows. 
Anomaly points are reported based on the parameters provided to the function. \\
In order to map dates against every anomaly points instead of array index, anomalyDatesSlopeBaseddetetion is called which provides (start\_date,end\_date,slope\_value) as final output. 
\chapter{Linear Regression}
\label{appendix:linearRegression}
\section{Introduction}

This technique is applied on two time series where one is independent variable 
and other is dependent variable. Let's say independent variable 
is "x" represented by series1 and "y" is dependent variable which is 
represented by series2, where y=f(x). \\
\\
So, in this method, given values of both variables at different points, i.e. 
given many pairs of (x,y), which are represented here by series1 and series2, 
this technique tries to find relation between x and y, i.e. it tries to find 
best suitable function y=f(x), which can best fit given data. Note that this 
function can only find linear relation between two variables, i.e. it can find 
relation such as y = mx + c, where "m" and "c" are some variables, which are 
found by this method, which can best represent these two series. \\
\\
After finding that function, for a given value of "x" one can predict, what 
should be ideal value of "y". So, this technique basically works on this 
principle. After finding that function, we again apply same function of the 
given series of "x" and try to predict corresponding series of "y" and see the 
relative difference between actual "y" series and predicted "y" series. If this 
relative difference is too high or too low or both (depending upon 
what user needs), we return those values as anomalies. To decide, whether value 
is too high or too low, we set up threshold. This threshold can be given by user 
or can be set automatically by using MAD test on the series generated by taking 
relative difference. Values beyond this threshold are reported as anomalies.

\section{Related Functions}

\subsection{linear\_regression(x\_series, y\_series, param = 0, 
default\_threshold = True, threshold = 0)}

This function takes two time series, x\_series and y\_series as input, where 
x\_series is series corresponding to "x" variable (independent variable) and 
y\_series is series corresponding to "y" variable (dependent variable, dependent 
on "x"). Given these two series, it first finds out best linear relationship 
between these two variables and as described in the above section, it finds 
relative difference between predicted and actual "y" series and the ones which 
are beyond threshold value are reported as anomaly. \\
\\
As described above, threshold value may be calculated by MAD test on relative 
difference values by keeping "default\_threshold" as "True", and if it is false, 
user will provide threshold value, by setting up "threshold" parameter above.\\
\\
Note that i'th value in y\_series should be corresponding to i'th value in the 
x\_series.

\begin{itemize}
 \item Input Parameters
 
 \begin{enumerate}
  \item x\_series \textit{(list)} : List of float values representing "x" 
variable (independent variable)
  \item y\_series \textit{(list)} : List of float values representing "y" 
variable (dependent variable)
  \item param \textit{(int, 1 or 0 or -1)} : \\
  		Defines what to be treated as anomaly depending on its value as 
follows: \\
        0: Values going out of range, both with positive and negative error \\
        1: Values with positive errors \\
        -1: Values with negative errors \\
        (Here error is relative difference crossing threshold value, 
positive error is relative difference which is positive and crossing positive 
threshold value and vice-versa).
  \item default\_threshold \textit{(boolean, True or False)} : If this is set as 
"True", than threshold will be calculated using MAD test, if False, than user 
given threshold value will be used.
  \item threshold \textit{(float)} : Here, user can provide threshold value if, 
default\_threshold is False.

 \end{enumerate}

 \item Output \textit{(Tuple)} : \\
 	returns Following tuple:
  (result,regression\_object) \\
  \\
  \\
  Where, "result" is list of tuples which are anomaly according to linear 
regression test of following format: \\
	\\  
  
(Index\_of\_Data\_Point,x\_value,y\_value,predicted\_y\_value, \\
difference\_between\_predicted\_and\_actual\_y\_value)
  \\
  \\
  "regression\_object" is an object of linear regression test, which represents 
y=f(x) = mx + c,  which can be used to regenerate predicted values for plotting 
graphs afterwards or for some other task.
  \\
  \\
  \textit{Format of using}: regression\_object.predict(x\_value), where x\_value 
is just one value, for which we need corresponding ideal "y" value.
 
\end{itemize}

\subsection{anomalies\_from\_linear\_regression(result\_of\_lr, any\_series)}

This function basically takes result of "linear\_regression" as input along with 
any series which is list of tuples of the form (date, value), and gives date to 
each anomaly. \\
\\
The result returned by "linear\_regression" function just provides index of data 
point, which is reported as anomaly. But we have time series, so we need to 
provide date, instead of index of data point. So, this function basically, 
attaches each anomaly with its date and returns it.

\begin{itemize}
 \item Input Parameters
 
 \begin{enumerate}
  \item result\_of\_lr \textit{(list)} : This is list of anomalies reported by 
"linear\_regression" function. Note that here we are just passing list of 
anomalies only and not the regression object, i.e. we are passing just first 
element of tuple returned by "linear\_regression" function.
  \item any\_series \textit{(list)} : Any list/series (x\_series or y\_series ) 
of tuples in the format (Date,Value), date will be used from this series to 
attach each anomaly with its corresponding date.
 \end{enumerate}

 \item Output \textit{(list)} : \\
 	Returns list of tuples of the following form: \\ 
 	
(date,x\_value,y\_value,predicted\_y\_value,
difference\_between\_predicted\_and\_actual\_y\_value)

\end{itemize}

\subsection{linear\_regressionMain(x\_series, y\_series, param = 0, 
default\_threshold = True, threshold = 0)}

This is main function of this anomaly detection technique. This function first 
calls "linear\_regression" function, gets list of anomalies. After that, it 
calls "anomalies\_from\_linear\_regression" function to attach date with each 
anomaly and than returns result.

\begin{itemize}
 \item Input Parameters
 
 \begin{enumerate}
  \item x\_series \textit{(list)} : List of tuples of the format (date,value) 
representing "x" variable (independent variable)
  \item y\_series \textit{(list)} : List of tuples of the format (date,value) 
representing "y" variable (dependent variable)
  \item param \textit{(int, 1 or 0 or -1)} : \\
  		Defines what to be treated as anomaly depending on its value as 
follows:\\
        0: Values going out of range, both with positive and negative error\\
        1: Values with positive errors\\
        -1: Values with negative errors\\
        (Here error is relative difference crossing threshold value, 
positive error is relative difference which is positive and crossing positive 
threshold value and vice-versa).
  \item default\_threshold \textit{(boolean, True or False)} : If this is set as 
"True", than threshold will be calculated using MAD test, if False, than user 
given threshold value will be used.
  \item threshold \textit{(float)} : Here, user can provide threshold value if, 
default\_threshold is False.

 \end{enumerate}

 \item Output \textit{(list)} : \\
 	Returns list of tuples of the form \\
 	
(start\_date,end\_date,difference\_between\_predicted\_and\_actual\_y\_value) \\
 	\\
 	Note that here, start\_date is equal to end\_date, as we are working 
day-wise in this technique, instead of any window.
 
\end{itemize}

\section{Description}

Putting all together, here is the summary:\\
\\
"linear\_regressionMain" is the main function of this technique, which calls 2 
other functions and returns result. First it calls, "linear\_regression" 
function, gets list of anomalies. After that, it calls 
"anomalies\_from\_linear\_regression" function to attach date with each anomaly 
and than returns result. 
\chapter{Graph Based Anomaly Detection Technique}

\section{Introduction}

This technique was introduced by [1]. We have used is R implementation given by 
authors of this book [2]. So, here by using python script, we will be just 
calling R script with appropriate arguments and will be using result provided 
by that script.

Graph based anomaly detection technique considers each day as a  node of a 
graph. Similar nodes are connected to each other by some weight. Similarity of 
nodes are calculated by making use of the values of that node i.e. value(s) of 
timeseries on that date. Based on this similarity, edge weights are also 
assigned. Than random walk algorithm is applied on this graph structure and 
connectivity value of each node is calculated. Graph nodes having the least 
connectivity values are reported as anomaly.

Note that previous techniques, like Window 
Correlation, Slope Based and Linear Regression techniques, can take only 2 time 
series as input. They also don't consider historical values, trend or 
seasonality. It just makes prediction on the given present data. Where as, this 
Graph based anomaly detection technique, can take multiple time series as input 
and also considers trends, seasonality as well, as explained in research paper 
[1].

So, here, we take multiple time series as input. Out of them, one will be 
dependent on rest of the others. We will call R script, it will print result in 
one csv file. We read that CSV file and return result. Note that here we do not 
have threshold value. We just give number of points with the least connectivity 
value and function returns them. If in future, one wants to add threshold value 
on connectivity than function can be modified according to that as well.

\section{Related Functions}

\subsection{graphBasedAnomalyCall(dependentVar, numberOfVals, 
timeSeriesFileNames)}

This function calls the R Script ``graphBasedAnomaly.R''. This function takes 
multiple time series as input, which are stored in files, whose name are stored 
in ``timeSeriesFileNames'' list. This time-series files are generated by us 
only. Out of these time series, one will be for dependent variable and others 
will be corresponding to independent variable. So variable, ``dependentVar'' 
represents which time series/variable is dependent.

this function executes R script and writes output to the file named 
``GraphBasedAnomalyOp.csv''.

\begin{itemize}
 \item Input Parameters
 
 \begin{enumerate}
  \item dependentVar \textit{(int)} : Index of the dependent variable, where 
dependentVar = function of independantVars
  \item numberOfVals \textit{(int)} : Each CSV contains how many values? That is 
each time series has how many values?
  \item timeSeriesFileNames \textit{(list)} : Name of the files to which series 
is stored. File should contain only series values.

 \end{enumerate}

 \item Output: This function does not generate any output. R Script will write 
output to CSV file as stated before.

 
\end{itemize}


\subsection{generateCSVsForGraphBasedAnomaly(lists, dateIndex, seriesIndex)}

In python code, we have time-series as a list. This list is list of tuples, in 
which first value of tuple is date and than we have more than one values in the 
same tuple, representing different time-series. For example, if we have 
test-case as onion, than for one city we have 3 time series along with date, 
which is rpresented as list of tuples of the the form (date, arrival, wholesale 
price, retail price). But, for R script, we need just time series 

\section{Description}

Putting all together, here is the summary:\\
\\
 
\chapter{Multivariate Time Series Anomaly Detection Technique}

\section{Introduction}

The method uses vector autoregressive framework for multivaraiate time-series analysis
in order to forecast values. The framework treats all the variables as symmetrical
and all the variables are modelled as if they influence each others equally.\\
\\
VAR generates forecast values for all the variables in recursive manner. Since VAR 
works on only stationary series, lag needs to be found so that the series could be
differenced in order to make them stationary.\\
\\
The code is implemented in R and python is used to call the R script with Appropriate 
arguments and process the intermeddiate results generated from the script. Forecast and 
vars library are used in R to implement VAR model for multiple time-series.\\
\\
All the interrelated time-series are passed to R script using a csv file. 60\% 
(This can be configured as per user need) of the passed data for every time-series 
is consumed for modelling the time-series. Rest of the time-series data is used to 
find the anomalies in the system by finding predicted values and range of higher and lower
predicted values.\\
\\
Multiple csv files (one for each varaiable) are generated as an output to the R script 
call which has actual values of variables along with the predicted ,lower and higher 
values of prediction. All the points which does not fall in the forecasted range and 
the percentage differnce between the actual and forecasted value breached threshold 
are reported as anomalies.\\
\\
Note that the threshold is computed with MAD test on the percentage of differnce between actual and forecasted value.
If in future, one wants to add threshold value than function can be modified according to that as well.\\
\\
Refer \cite{var} for more detailed information.

\section{Related Functions}

\subsection{MultivariateAnomaly(fileName,hd, paramCount,fileStart)}

This function is inside R script which takes a fileName containing all the related 
time-series and generate output files (one file for every component) containing predicted, actual,
lower and higher forecasted values. \\
\\
The name of output files starts with fileStart appended with
the sequence number of variable. Like if ``RetailWsArrival'' is passed as fileStart and there are 
two variables or series in input file. The names of generated file will be : RetailWsArrival1.csv and 
RetailWsArrival2.csv

\begin{itemize}
 \item Input Parameters
 
 \begin{enumerate}
  \item fileName \textit{(string)} : Name of the file which contains all the interrelated time series for model
  \item hd \textit{(boolean)} : Whether the CSV file contains header for columns or not 
  \item paramCount \textit{(int)} : Number of variables in file 
  \item fileStart \textit{(string)} : Prefix for the name of output files to be generated
  
 \end{enumerate}

 \item Output: This function will write output to CSV file as stated before.

\end{itemize}

\subsection{multivaraiateAnalysis(args)}

This function calls the R script through python passing args as the input to the R script.

\begin{itemize}
 \item Input Parameters
 
 \begin{enumerate}
  \item args \textit{(list)} : list of strings which serve as input to R script 

 \end{enumerate}

 \item Output: No output.

\end{itemize}

\subsection{csvTransform(filePath,startDate)}

This function segregate the anomalies from all the other points for which the forcast was 
generated by R script function named ``MultivariateAnomaly``.

\begin{itemize}
 \item Input Parameters
 
 \begin{enumerate}
  \item filePath \textit{(string)} : Path of output file generated by R script ''MultivariateAnomaly`` which contains actual, forecasted, lower and higher values  
  \item startDate \textit{(string)} : The date from which the forecast was generated by R script
 \end{enumerate}

 \item Output \textit{(list)}: \\
  returns list of tuples of the form: \\
  (start\_date, end\_date, percDiff) \\
  Where percDiff is the percentage differnce between the actual and forecasted value.
  Note that, here start\_date will be same as end\_date, as this function 
returns results day-wise.
 

\end{itemize}


\section{Description}

Putting all together, here is the summary:\\
\\
''MultivariateAnomaly`` is the R Script function which is called by  python function ''multivaraiateAnalysis``.
It generates forecasted values based on the model generated by 60\% of the input data.
Lastly, csvTransform processes the generated file in order to report anomalies which are falling outside the range of lower
and higher forecast value and breach the threshold calculated using MAD test on percentage of differnce between actual amd 
forecasted value.
\chapter{Utility}

\section{Introduction}
         
Here, we explain some related funtions which are used by stated anomaly 
detection techniques. Note that some of these functions can be used to process 
the results of the anomaly techniques.

\subsection{Functions used by Anomaly detection techniques}

\begin{itemize}
 \item convertListToFloat(li) \\
	Given list of elements, this function type casts all the elements to 
float type.
	
	\begin{itemize}
	  \item Input Parameters
	  
	  \begin{enumerate}
	    \item li \textit{(int)} : list of elements	    
	  \end{enumerate}

	  \item Output \textit{(list)}: \\
	  Returns list of elements after converting each element into float 

	  \end{itemize}
    
	
	
	
	
 \item getColumnFromListOfTuples(lstTuples,i) \\
 
    This function returns i'th element of all tuples as a list.
    
    \begin{itemize}
	  \item Input Parameters
	  
	  \begin{enumerate}
	    \item lstTuples \textit{(int)} : list of tuples. Tuples has no fixed 
format
	    \item i \textit{(int)} : index of which tuple element to return, 
index starting from zero
	  \end{enumerate}

	  \item Output \textit{(list)}: \\
	  Returns list of elements after fetching i'th element from each tuple.
	  
    \end{itemize}
 
 
 \item findAverageTimeSeries(timeSeriesCollection) \\
 
	It takes 2D list of element, where each element of timeSeriesCollection 
is one time series. It returns average of all time series. (first element of 
resltant time series will be average of first element of all time series) \\
	
	For example let,\\
	timeSeriesCollection: [ \newline
	    [1,2,3], \# Timeseries 1 \newline
	    [4,5,6], \# Timeseries 2 \newline
	    [7,8,9] \# Timeseries 3 \newline
	] \newline
	\\
	This function will return,\newline
	[4,5,6] \newline
	
	\begin{itemize}
	  \item Input Parameters
	  
	  \begin{enumerate}
	    \item timeSeriesCollection \textit{(list)} : 2D array of float 
elements.
	  \end{enumerate}

	  \item Output \textit{(list)}: \\
	  Returns list after taking average of all time series.
	  
	\end{itemize}
 
 
 
 \item writeToCSV(lstData,fileName) \\
 \item concateLists(lstData) \\
 \item cleanArray(array) \\
 
	This function removes ``nan'' (Not a number values) from the list.
	
	\begin{itemize}
	  \item Input Parameters
	  
	  \begin{enumerate}
	    \item array \textit{(list)} : list of float type elements
elements.
	  \end{enumerate}

	  \item Output \textit{(list)}: \\
	  Returns list of float elements after removing ``nan'' elements.
	  
	\end{itemize}
 
 
 \item MADThreshold(array) \\
 \item smoothArray(array, alpha = 2.0/15.0) \\
 \item csv2array(filePath) \\
 \item getColumn(array, column\_number) \\
 \item formatCSV2Array(z) \\

 \item csvTransform(filePath,startDate) \\

\end{itemize}


\subsection{Functions used to process results}

\begin{itemize}
   
   
   \item intersection(numOfResults, list1, resultOf1, list2, resultOf2, list3 = 
[], resultOf3="linear\_regression", list4=[], resultOf4="graph\_based", 
list5=[], resultOf5="spike\_detection" , list6=[], resultOf6="multiple\_arima") 
\\
  This function is used to take intersection of results of multiple methods (2 
or more), which are passed as list here. This function requires minimum of 5 
arguments. 
   
   \begin{itemize}
 \item Input Parameters
 
 \begin{enumerate}
  \item numOfResults \textit{(int)} : number of lists you are passing, minimum 2
  \item $list_i$ \textit{(list)} : list represnting result of the i'th 
algorithm. Note that this is list of tuples of the format, \\
  (startDate, endDate, value)
  \item $resultOf_i$ \textit{(string)}: this variable states, $list_i$ is of 
which algorithm, it can be from following: \\
            (slope\_based, linear\_regression, graph\_based, spike\_detection, 
multiple\_arima)

 \end{enumerate}

 \item Output \textit{(list)}: \\
 This function returns intersection of all lists. Returned value is list of 
tuples of the form: \\
 (date, correlation value, slope\_based value, linear\_regression value, 
graph\_based value, spike\_detection value, multiple\_arima value)

 \end{itemize}
   
   
   
   
   \item intersectionOfFinalResults(list1, list2) \\
  
  This function takes intersection of 2 lists (where each list is list of 
tuple) and returns result. Note that these input lists are generated as a output 
of ``intersection'' method.
 \begin{itemize}
 \item Input Parameters
 
 \begin{enumerate}
  \item list1 \textit{(list)} : List of tuples of the format: \\
  (date, correlation value, slope\_based value, linear\_regression value, 
graph\_based value, spike\_detection value, multiple\_arima value)
  \item list2 \textit{(list)} :  List of tuples of the format:  \\
  (date, correlation value, slope\_based value, linear\_regression value, 
graph\_based value, spike\_detection value, multiple\_arima value)

 \end{enumerate}

 \item Output \textit{(list)}: \\
 This function returns intersection of list1 and list2. Returned value is list 
of tuples of the form: \\
 (date, correlation value, slope\_based value, linear\_regression value, 
graph\_based value, spike\_detection value, multiple\_arima value)

 \end{itemize}
 
 
 
 \item unionOfFinalResults(list1, list2) \\
 This function takes union of 2 lists (where each list is list of tuple) and 
returns result. Note that these input lists are generated as a output of 
``intersection'' method.
 \begin{itemize}
 \item Input Parameters
 
 \begin{enumerate}
  \item list1 \textit{(list)} : List of tuples of the format: \\
  (date, correlation value, slope\_based value, linear\_regression value, 
graph\_based value, spike\_detection value, multiple\_arima value)
  \item list2 \textit{(list)} :  List of tuples of the format:  \\
  (date, correlation value, slope\_based value, linear\_regression value, 
graph\_based value, spike\_detection value, multiple\_arima value)

 \end{enumerate}

 \item Output \textit{(list)}: \\
 This function returns union of list1 and list2. Returned value is list of 
tuples of the form: \\
 (date, correlation value, slope\_based value, linear\_regression value, 
graph\_based value, spike\_detection value, multiple\_arima value)

 \end{itemize}
 
 
 
 
  \item mergeDates(li) \\
  
  
  This function merges overlapping time period. The list, which it takes as 
input, ``li'', is list of tuples, of the format, \\
  (startDate, endDate, Value). \\
  \\
  So if, we have overlapping period or two time periods are adjacent, for 
example if one tuple is (1-1-2015, 15-1-2015, 5) and other tuple is (15-1-2015, 
30-1-2015, 6), than this function will produce output as, 
(1-1-2015, 30-1-2015, 5).
  
  \begin{itemize}
 \item Input Parameters
 
 \begin{enumerate}
  \item li \textit{(list)} : List of tuples of the format: \\
  (startDate, endDate, Value) \\
  Note that here startDate and endDate, are of type \textit{datetime} and Value 
is of type \textit{float}.
 \end{enumerate}

 \item Output \textit{(list)}: \\
 This function returns list of tuples of the form: \\
 (startDate, endDate, Value)

 \end{itemize}
  
  
  
  
 \item resultOfOneMethod(array) \\
 
 
 This function just converts format of the result list. Usually, anomaly 
detection methods returns list of tuples of the format, \\ 
 (startDate, endDate, Value) \\
 this function will convert it to list of tuples of the format, \\
 (date, value) \\
 \\
 Basically, all the dates between startDate and endDate will be added to the 
result list.
 
 \begin{itemize}
 \item Input Parameters
 
 \begin{enumerate}
  \item array \textit{(list)} : List of tuples of the format: \\
  (startDate, endDate, Value) \\
  Note that here startDate and endDate, are of type \textit{datetime} and Value 
is of type \textit{float}.
 \end{enumerate}

 \item Output \textit{(list)}: \\
 This function returns list of tuples of the form: \\
 (date, Value)
  Note that here date is of type \textit{datetime} and Value is of type 
\textit{float}.
 \end{itemize}
 
 
 
 
 
 
 
 
 
 
 
 
 
 
 
 
 
 
 
\end{itemize}


 

\bibliographystyle{unsrt}
\bibliography{biblio}
\end{document}
