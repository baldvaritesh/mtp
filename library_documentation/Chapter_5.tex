\chapter{Multivariate Time Series Anomaly Detection Technique}

\section{Introduction}

The method uses vector autoregressive framework for multivaraiate time-series analysis
in order to forecast values. The framework treats all the variables as symmetrical
and all the variables are modelled as if they influence each others equally.

VAR generates forecast values for all the variables in recursive manner. Since VAR 
works on only stationary series, lag needs to be found so that the series could be
differenced in order to make them stationary.

The code is implemented in R and python is used to call the R script with Appropriate 
arguments and process the intermeddiate results generated from the script. Forecast and 
vars library are used in R to implement VAR model for multiple time-series.

All the interrelated time-series are passed to R script using a csv file. 60\% 
(This can be configured as per user need) of the passed data for every time-series 
is consumed for modelling the time-series. Rest of the time-series data is used to 
find the anomalies in the system by finding predicted values and range of higher and lower
predicted values.

Multiple csv files (one for each varaiable) are generated as an output to the R script 
call which has actual values of variables along with the predicted ,lower and higher 
values of prediction. All the points which does not fall in the forecasted range and 
the percentage differnce between the actual and forecasted value breached threshold 
are reported as anomalies.

Note that the threshold is computed with MAD test on the percentage of differnce between actual and forecasted value.
If in future, one wants to add threshold value than function can be modified according to that as well.

Refer \cite{var} for more detailed information.

\section{Related Functions}

\subsection{MultivariateAnomaly(fileName,hd, paramCount,fileStart)}

This function is inside R script which takes a fileName containing all the related 
time-series and generate output files (one file for every component) containing predicted, actual,
lower and higher forecasted values. 

The name of output files starts with fileStart appended with
the sequence number of variable. Like if ``RetailWsArrival'' is passed as fileStart and there are 
two variables or series in input file. The names of generated file will be : RetailWsArrival1.csv and 
RetailWsArrival2.csv

\begin{itemize}
 \item Input Parameters
 
 \begin{enumerate}
  \item fileName \textit{(string)} : Name of the file which contains all the interrelated time series for model
  \item hd \textit{(boolean)} : Whether the CSV file contains header for columns or not 
  \item paramCount \textit{(int)} : Number of variables in file 
  \item fileStart \textit{(string)} : Prefix for the name of output files to be generated
  
 \end{enumerate}

 \item Output: This function will write output to CSV file as stated before.

\end{itemize}

\subsection{multivaraiateAnalysis(args)}

This function calls the R script through python passing args as the input to the R script.

\begin{itemize}
 \item Input Parameters
 
 \begin{enumerate}
  \item args \textit{(list)} : list of strings which serve as input to R script 

 \end{enumerate}

 \item Output: No output.

\end{itemize}

\subsection{csvTransform(filePath,startDate)}

This function segregate the anomalies from all the other points for which the forcast was 
generated by R script function named ``MultivariateAnomaly``.

\begin{itemize}
 \item Input Parameters
 
 \begin{enumerate}
  \item filePath \textit{(string)} : Path of output file generated by R script ''MultivariateAnomaly`` which contains actual, forecasted, lower and higher values  
  \item startDate \textit{(string)} : The date from which the forecast was generated by R script
 \end{enumerate}

 \item Output \textit{(list)}: \\
  returns list of tuples of the form: \\
  (start\_date, end\_date, percDiff) \\
  Where percDiff is the percentage differnce between the actual and forecasted value.
  Note that, here start\_date will be same as end\_date, as this function 
returns results day-wise.
 

\end{itemize}


\section{Description}

Putting all together, here is the summary:\\
\\
''MultivariateAnomaly`` is the R Script function which is called by  python function ''multivaraiateAnalysis``.
It generates forecasted values based on the model generated by 60\% of the input data.
Lastly, csvTransform processes the generated file in order to report anomalies which are falling outside the range of lower
and higher forecast value and breach the threshold calculated using MAD test on percentage of differnce between actual amd 
forecasted value.