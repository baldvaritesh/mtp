\documentclass[a4paper,10pt]{article}
\usepackage[utf8]{inputenc}
\usepackage{graphicx}    
\usepackage{color}
%\usepackage{epsfig}   
\usepackage[font=footnotesize]{subfig}
\usepackage{float}
\usepackage{fancyhdr}                              
\usepackage{makeidx}
\usepackage[nottoc,notlot,notlof]{tocbibind}     
\usepackage{supertabular}
\usepackage{array}              
\usepackage{setspace} 
\usepackage{enumerate}
\usepackage{rotating}
\usepackage{moreverb}
\usepackage{multirow}
\usepackage{amsmath}
\usepackage{amsthm}
\usepackage{amssymb}
\usepackage{captcont}
\usepackage{verbatim}
\usepackage{titlesec}
\usepackage{url}
\usepackage{hyperref}
\usepackage{lipsum}
\usepackage{tikz}
\usepackage{pgf-pie}
\usepackage{pgfplots}
\usepackage{array}
\usepackage{booktabs}
\usepackage{blindtext}
\usepackage{tabularx}
\usepackage{pbox}
%\usepackage[utf8]{inputenc}
\usepackage{commath}

\newcolumntype{L}[1]{>{\raggedright\let\newline\\\arraybackslash\hspace{0pt}}m{#1}}
%\usepackage[margin=1in]{geometry}

\title{How To}

\begin{document}
\maketitle
\section{Introduction}
Our project code can be found on Git Hub over here:\\
\\
https://github.com/kapilthakkar72/mtp\\
\\
Each folder has ReadMe file in it stating the files present in it.

\begin{itemize}

\item To refer our work, see our Thesis here:
	\begin{itemize}
		\item Thesis\_of\_both\_members/Kapil\_Thakkar\_2014MCS2124.pdf
		\item Thesis\_of\_both\_members/Reshma\_Kumari\_2014MCS2134.pdf
	\end{itemize}
	
\item To refer code/library refer folder:
	\begin{itemize}
		\item library/
	\end{itemize}

\item To see detailed documentation of library functions refer document:
	\begin{itemize}
		\item library\_documentation/Anomaly Detection.pdf
	\end{itemize}
	

\item To refer data used by us, refer folder:
	\begin{itemize}
		\item database\_backup/
	\end{itemize}

\item To see our results refer folder:
	\begin{itemize}
		\item analysis/
	\end{itemize}
  Documentation can be found here:
  	\begin{itemize}
		\item analysis/resultDoc/results.pdf
	\end{itemize}
  
  	
\item Work done to collect news articles, fetching links from html search pages, fetching text and related data using AlchemyAPI and Diffbot can be found here:
	\begin{itemize}
		\item newsArticleWork/
	\end{itemize}

\item For graph based anomaly, R packages are present in folder:
	\begin{itemize}
		\item mtp/library/R\_Packages/
	\end{itemize}
	
	
\item Matlab Script for mapping mandis to center using voronoi diagram can be found here:
	\begin{itemize}
		\item mtp/matlabScript/clustering.m
	\end{itemize}
	
\item Crawlers written to fetch onion data are present here:
	\begin{itemize}
		\item Wholesale Price and Arrival: mtp/oniondataCollection/WholeSalePriceCrawler
		\item Retail Price: mtp/oniondataCollection/RetailPriceCrawer
		\item Location of Mandis and Centers: mtp/oniondataCollection/LocationScript
		\item Required JAR files: mtp/oniondataCollection/jarFiles
	\end{itemize}

\item Final Presentation is present here:
	\begin{itemize}
		\item thesisPresentation/Thesis - Time series analysis.pptx
	\end{itemize}

\end{itemize}

To see the detailed information refer ReadMe file in each folder.

\section{Other Details}

\begin{itemize}
	\item Library scripts are written in Python and R. So it is necessary that both are installed on system. 
	\item Packages required to run code can easily be downloaded from repos of corresponding language. Additional R packages required are stated in library folder (Refer ReadMe of library folder).
	\item Required jar files to run crawlers and location script are also provided.
	\item Documentation of Code executed by us to do analysis and detailed analysis of functions used is present in \\
			library\_documentation/Anomaly Detection.pdf
	\item Details about code and data can be found in the ReadMe file in the corresponding folder.
\end{itemize}

\end{document}   