\chapter{Slope Based Detection}

\section{Introduction}

The method works on two time-series. It finds the ratio of steepness 
at two different points in the time-series. Let's say we have two 
time series as series1 and series2. So in this method, we first find 
the rate of change in the time-series values for both time-series followed 
by taking ratio of these rate of change. i.e  suppose we have two points on 
time-series 1 as $y_{11}$ and $y_{12}$ and on time-series 2 as $y_{21}$ and $y_{22}$. 
Rate of change between these points is calculated as following
 $$S_1=\frac{y_{12}-y_{11}}{y_{11}}$$
 ,
  $$S_2=\frac{y_{22}-y_{21}}{y_{21}}$$
 Ratio of steepness ( \textbf{rs} ) is calculated as
 $$rs = \frac{S_1}{S_2}$$

The \textbf{rs} is calculated between first and last point of every window of
size ``w'' provided as input. \\
\\
Now, we have rate of change in the steepness(\textbf{rs}) for every window. 
The outliers are detected by the threshold value provided by user or if not, 
then threshold is computed using MAD test on the all \textbf{rs} calculated above.\\
\\
If the two time-series are expected to move in tandem, then all the points 
with \textbf{rs} greater than threshold are reported whereas if the two time 
series should not move in tandem then all the points with \textbf{rs} less than 
threshold are reported. 

\section{Related Functions}

\subsection{slopeBasedDetection(series1,smoothed1,\\series2,smoothed2, next\_val\_to\_consider,\\ default\_threshold, threshold, what\_to\_consider)}

This function smooths the provided time series data using exponential moving average(if needed) and 
calculates \textbf{rs} for first and last point of every window of size next{\_}val{\_}to{\_}consider.

% Since, we calculate rs at regular interval, to avoid sudden spike or steepness in result, we prefer smoothed time-series.

\begin{itemize}
 \item Input Parameters
 
 \begin{enumerate}
  \item series1 \textit{(list)} : Input series 1 as a list of float values
  \item smoothed1 \textit{(boolean)} : Whether series1 is smoothed or not? If not (value of this parameter is \textit{False} ) smoothing will be done
  \item series2 \textit{(list)} : Input series 2 as a list of float values
  \item smoothed2 \textit{(boolean)} : Whether series2 is smoothed or not? If not (value of this parameter is \textit{False} ) smoothing will be done
  \item next{\_}val{\_}to{\_}consider \textit{(int)} : indicates the size of window or next point in time-series to calculate slope of steepness. Default is 7 days.
  \item default{\_}threshold \textit{(int)} : Whether to consider default threshold or not. If \textit{True}, the threshold is calculated using MAD Test.
  \item threshold \textit{(int)} : Threshold value to consider if default\_threshold is set to \textit{False}
  \item what{\_}to{\_}consider \textit{(int)} : Can be either 1,0 or -1. If the series are supposed to move in tandem, 1 is set otherwise -1 is set. In case we don't know the correlation between two, 0 is set.
  \end{enumerate}

 \item Output \textit{(list)} : \\
  Returns list of tuples of the form \\
  \center{(first,second,slope\_value)}\\
  \begin{flushleft}
 where first and second are the array index of passed series for which the \textbf{rs} is calculated.
 \end{flushleft}
\end{itemize}

\subsection{anomalyDatesSlopeBaseddetetion\\(slopeBasedResult,any\_series)}

This function basically takes result of ”slopeBasedDetection” as input along with
any series which is list of tuples of the form (date, value), and gives date to each
anomaly.\\

The result returned by ”slopeBasedDetection” function just provides index of
data point, which is reported as anomaly. But we have time series, so we need
to provide date, instead of index of data point. So, this function basically,
attaches each anomaly with its date and returns it.\\

\begin{itemize}
 \item Input Parameters
 
 \begin{enumerate}
  \item slopeBasedResult \textit{(list)} : This is list of anomalies reported by ”slopeBasedDetection” function.
  \item any\_series \textit{(list)} : Any list/series of tuples in the format (Date,Value), date will be used from this series to find date against each anomaly.
 \end{enumerate}

 \item Output \textit{(list)} : \\
 	Returns list of tuples of the following form: \\ 
 	(start\_date,end\_date,slope\_value)

\end{itemize}


\subsection{slopeBased(series1,smoothed1,series2,smoothed2,\\next\_val\_to\_consider, default\_threshold, \\threshold, what\_to\_consider)}

This is main function of this anomaly detection technique. This function first calls "slopeBasedDetection" function, gets list of anomalies. 
After that, it calls "anomalyDatesSlopeBaseddetetion" function to attach date with each anomaly and than returns result.

\begin{itemize}
 \item Input Parameters
 
 \begin{enumerate}
  \item series1 \textit{(list)} : Input series 1 as a list of tuples of the forms (date, value)
  \item smoothed1 \textit{(boolean)} : Whether series1 is smoothed or not? If not (value of this parameter is \textit{False} ) smoothing will be done
  \item series2 \textit{(list)} : Input series 2 as a list of tuples of the forms (date, value)
  \item smoothed2 \textit{(boolean)} : Whether series2 is smoothed or not? If not (value of this parameter is \textit{False} ) smoothing will be done
  \item next{\_}val{\_}to{\_}consider \textit{(int)} : indicates the size of window or next point in time-series to calculate slope of steepness. Default is 7 days.
  \item default{\_}threshold \textit{(int)} : Whether to consider default threshold or not. If \textit{True}, the threshold is calculated using MAD Test.
  \item threshold \textit{(int)} : Threshold value to consider if default\_threshold is set to \textit{False}
  \item what{\_}to{\_}consider \textit{(int)} : Can be either 1,0 or -1. If the series are supposed to move in tandem, 1 is set otherwise -1 is set. In case we don't know the correlation between two, 0 is set.
  \end{enumerate}

 \item Output \textit{(list)} : \\
 	Returns list of tuples of the following form: \\ 
 	(start\_date,end\_date,slope\_value)\\
 	where, start\_date and end\_date are points on which the steep was computed along with the slope\_value which was spotted as outlier.
 
\end{itemize}

\section{Description}

slopeBased function is called to find anomalies based on the rate of change in value. It calls slopeBasedDetection method to compute the slopes between points in windows. 
Anomaly points are reported based on the parameters provided to the function. \\
\\
In order to map dates against every anomaly points instead of array index, anomalyDatesSlopeBaseddetetion is called which provides (start\_date,end\_date,slope\_value) as final output.