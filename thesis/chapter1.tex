\chapter{Introduction}


\section{Motivation}

Supply demand imbalance, natural calamities etc. may not always be the reason behind the rise in the price of a commodity. ​It may be a consequence of artificial supply deficit planned intelligently by traders’ nexus for profiteering through manipulation of supply of commodity and hence indirectly controlling their prices. ​Our attempt is to locate such hikes in prices which seem suspicious (we call them anomalies).​ To detect and analyse the characteristics of anomalies in the prices of commodities. Currently we have considered the case of onion and based on that we have developed one library which has multiple functions to detect anomalies in the time series.


\section{Objective}

Our objective is that we need to highlight anomalies which may be an indicator of illegal market manipulation act by traders nexus in the provided time series. For this purpose we have created library with set of functionalities to detect anomalies in the given input time series. Anomalies will be reported based on the hypothesis stated by the user. For different scenarios user can pass appropriate parameters to functions and functions will report anomalies to user based on that.


\section{Relevance of Project}

Anomaly detection techniques help to explore situations which might be different from the expected behaviour and could reveal interesting facts. It is used in many areas which are explained in detail in next chapter. The project aims to raise potential red flags for days which are suspect of illicit market manipulation activities by set of traders. This type of monitoring system may help people monitor and hence control these illicit market manipulations better.
For example, unnecessary hike in the prices may be due to wrong government policies, loopholes in the supply chain of commodity, intention of profiteering by traders etc. So our project will help journalist or end user interested in detecting such abnormal behaviour.
