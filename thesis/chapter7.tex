\chapter{Conclusion and Future Work}

\section{Conclusion}

In Chapter 5, we presented overall system result. Comparing anomalies reported by system with the news articles present, results were quite good. Anomalies reported by system were more as compared to news articles. That might be because we have national news sources, which reports only big crisis in the news. When we analysed results produced by system, then we found them justifiable.

\section{Future Work}

This project can further be extended by adding new methods for various hypothesis. One such method is Spike Detection, which can be used for Hypothesis 2, to enhance the results. First we can generate series of relative difference between retail price and wholesale price. Then we apply spike detection method over it. If this difference becomes very large in the short duration of time, then it can be reported as anomaly. Reason to report this as anomaly is that there exists few news articles which reports such type of behaviour as anomaly. Apart from this other methods can also be explored.\\
\\
One can also consider value chain of any product, like let's say car. Then price of various components in this chain starting from raw material, raw parts and final product price, etc can be collected and one can find if there exists any anomaly at any point of time if price of final product goes up.\\
\\
MAD threshold has been used to calculate default threshold value. One can try other methods to calculate default threshold value. System can also be automated to select different parameters for library functions and can report results which are best.\\
\\
One can also extend analysis on regional basis by considering regional news papers. Here, we have considered only national news papers, but local news sources may provide some more insights into results.
