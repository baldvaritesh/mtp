\chapter{Conclusion}

Often rise in the price of commodity are considered as a result of unseasonal rainfall, increased demand, unrealistic government policies, supply deficit etc. But supply chain deficits and natural calamities can not only be held responsible for every hike in price of commodity. Sometimes the hike in prices are man-made with the intention of earning more profits with illicit means like excess hoarding, manual creation of supply crunch etc. As we have observed in the previous chapter as well that for the Mumbai centre, 32\% of the news articles state traders' nexus as reason for price hike of onions.\\
\\
These manual interventions to the supply chain of commodity for sake of earning more profit can be found by analysing time series data of dependent factors since these have distinguishing characteristics. These characteristics can be found by various statistical techniques.\\
\\
So, for given time series first one needs to understand normal behavior before going for detection of anomalies. Time series may have periodicity, seasonality, trends or may have complete randomness. So method to detect anomaly should be designed in a such a way that functionalities covered by it does not miss any type of time series. So, here we have tried to build up a system keeping multiple functionalities for one hypothesis, such that it can cover multiple aspects of any time series. Results produced by this are quite justifiable and acceptable.\\
\\
This project can further be extended by adding new methods for various hypothesis. One such method is Spike Detection, which can be used for Hypothesis 2, to enhance the results. First we can generate series of relative difference between retail price and wholesale price. Then we apply spike detection method over it. If this difference becomes very large in the short duration of time, then it can be reported as anomaly. Reason to report this as anomaly is that there exists few news articles which reports such type of behaviour as anomaly.\\
\\
One can also consider value chain of any product, like let's say car. Then price of various components in this chain starting from raw material, raw parts and final product price, etc can be collected and one can find if there exists any anomaly at any point of time if price of final product goes up.\\
