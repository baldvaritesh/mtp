\chapter{Linear Regression}

\section{Introduction}

This technique is applied on two time series where one is independent variable 
and other is dependent variable. Let's say independent variable 
is "x" represented by series1 and "y" is dependent variable which is 
represented by series2, where y=f(x). \\
\\
So, in this method, given values of both variables at different points, i.e. 
given many pairs of (x,y), which are represented here by series1 and series2, 
this technique tries to find relation between x and y, i.e. it tries to find 
best suitable function y=f(x), which can best fit given data. Note that this 
function can only find linear relation between two variables, i.e. it can find 
relation such as y = mx + c, where "m" and "c" are some variables, which are 
found by this method, which can best represent these two series. \\
\\
After finding that function, for a given value of "x" one can predict, what 
should be ideal value of "y". So, this technique basically works on this 
principle. After finding that function, we again apply same function of the 
given series of "x" and try to predict corresponding series of "y" and see the 
relative difference between actual "y" series and predicted "y" series. If this 
relative difference is too high or too low or both (depending upon 
what user needs), we return those values as anomalies. To decide, whether value 
is too high or too low, we set up threshold. This threshold can be given by user 
or can be set automatically by using MAD test on the series generated by taking 
relative difference. Values beyond this threshold are reported as anomalies.

\section{Related Functions}

\subsection{linear\_regression(x\_series, y\_series, param = 0, 
default\_threshold = True, threshold = 0)}

This function takes two time series, x\_series and y\_series as input, where 
x\_series is series corresponding to "x" variable (independent variable) and 
y\_series is series corresponding to "y" variable (dependent variable, dependent 
on "x"). Given these two series, it first finds out best linear relationship 
between these two variables and as described in the above section, it finds 
relative difference between predicted and actual "y" series and the ones which 
are beyond threshold value are reported as anomaly. \\
\\
As described above, threshold value may be calculated by MAD test on relative 
difference values by keeping "default\_threshold" as "True", and if it is false, 
user will provide threshold value, by setting up "threshold" parameter above.\\
\\
Note that i'th value in y\_series should be corresponding to i'th value in the 
x\_series.

\begin{itemize}
 \item Input Parameters
 
 \begin{enumerate}
  \item x\_series \textit{(list)} : List of float values representing "x" 
variable (independent variable)
  \item y\_series \textit{(list)} : List of float values representing "y" 
variable (dependent variable)
  \item param \textit{(int, 1 or 0 or -1)} : \\
  		Defines what to be treated as anomaly depending on its value as 
follows: \\
        0: Values going out of range, both with positive and negative error \\
        1: Values with positive errors \\
        -1: Values with negative errors \\
        (Here error is relative difference crossing threshold value, 
positive error is relative difference which is positive and crossing positive 
threshold value and vice-versa).
  \item default\_threshold \textit{(boolean, True or False)} : If this is set as 
"True", than threshold will be calculated using MAD test, if False, than user 
given threshold value will be used.
  \item threshold \textit{(float)} : Here, user can provide threshold value if, 
default\_threshold is False.

 \end{enumerate}

 \item Output \textit{(Tuple)} : \\
 	returns Following tuple:
  (result,regression\_object) \\
  \\
  \\
  Where, "result" is list of tuples which are anomaly according to linear 
regression test of following format: \\
	\\  
  
(Index\_of\_Data\_Point,x\_value,y\_value,predicted\_y\_value, \\
difference\_between\_predicted\_and\_actual\_y\_value)
  \\
  \\
  "regression\_object" is an object of linear regression test, which represents 
y=f(x) = mx + c,  which can be used to regenerate predicted values for plotting 
graphs afterwards or for some other task.
  \\
  \\
  \textit{Format of using}: regression\_object.predict(x\_value), where x\_value 
is just one value, for which we need corresponding ideal "y" value.
 
\end{itemize}

\subsection{anomalies\_from\_linear\_regression(result\_of\_lr, any\_series)}

This function basically takes result of "linear\_regression" as input along with 
any series which is list of tuples of the form (date, value), and gives date to 
each anomaly. \\
\\
The result returned by "linear\_regression" function just provides index of data 
point, which is reported as anomaly. But we have time series, so we need to 
provide date, instead of index of data point. So, this function basically, 
attaches each anomaly with its date and returns it.

\begin{itemize}
 \item Input Parameters
 
 \begin{enumerate}
  \item result\_of\_lr \textit{(list)} : This is list of anomalies reported by 
"linear\_regression" function. Note that here we are just passing list of 
anomalies only and not the regression object, i.e. we are passing just first 
element of tuple returned by "linear\_regression" function.
  \item any\_series \textit{(list)} : Any list/series (x\_series or y\_series ) 
of tuples in the format (Date,Value), date will be used from this series to 
attach each anomaly with its corresponding date.
 \end{enumerate}

 \item Output \textit{(list)} : \\
 	Returns list of tuples of the following form: \\ 
 	
(date,x\_value,y\_value,predicted\_y\_value,
difference\_between\_predicted\_and\_actual\_y\_value)

\end{itemize}

\subsection{linear\_regressionMain(x\_series, y\_series, param = 0, 
default\_threshold = True, threshold = 0)}

This is main function of this anomaly detection technique. This function first 
calls "linear\_regression" function, gets list of anomalies. After that, it 
calls "anomalies\_from\_linear\_regression" function to attach date with each 
anomaly and than returns result.

\begin{itemize}
 \item Input Parameters
 
 \begin{enumerate}
  \item x\_series \textit{(list)} : List of tuples of the format (date,value) 
representing "x" variable (independent variable)
  \item y\_series \textit{(list)} : List of tuples of the format (date,value) 
representing "y" variable (dependent variable)
  \item param \textit{(int, 1 or 0 or -1)} : \\
  		Defines what to be treated as anomaly depending on its value as 
follows:\\
        0: Values going out of range, both with positive and negative error\\
        1: Values with positive errors\\
        -1: Values with negative errors\\
        (Here error is relative difference crossing threshold value, 
positive error is relative difference which is positive and crossing positive 
threshold value and vice-versa).
  \item default\_threshold \textit{(boolean, True or False)} : If this is set as 
"True", than threshold will be calculated using MAD test, if False, than user 
given threshold value will be used.
  \item threshold \textit{(float)} : Here, user can provide threshold value if, 
default\_threshold is False.

 \end{enumerate}

 \item Output \textit{(list)} : \\
 	Returns list of tuples of the form \\
 	
(start\_date,end\_date,difference\_between\_predicted\_and\_actual\_y\_value) \\
 	\\
 	Note that here, start\_date is equal to end\_date, as we are working 
day-wise in this technique, instead of any window.
 
\end{itemize}

\section{Description}

Putting all together, here is the summary:\\
\\
"linear\_regressionMain" is the main function of this technique, which calls 2 
other functions and returns result. First it calls, "linear\_regression" 
function, gets list of anomalies. After that, it calls 
"anomalies\_from\_linear\_regression" function to attach date with each anomaly 
and than returns result.