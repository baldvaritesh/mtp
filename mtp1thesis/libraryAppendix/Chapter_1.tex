\chapter{Window Based Correlation}

\section{Introduction}

This technique is basically applied on two time-series. Let's say we have two 
time series as series1 and series2. So, in this method, we first find 
correlation at various lags between these two time series. User can specify 
minimum and maximum lag to consider. So, for each value between minimum and maximum lag, we find correlation values.\\
\\
After finding correlation values at all lags, we consider that lag at which 
correlation value is higher, among all previously calculated correlation values, 
at all lags. Let's say that lag be ``x''. So, depending upon that ``x'', we 
shift series1 or series2. If ``x'' is positive, we move series2 by ``x'' units 
and if it is negative than we shift series1 by \abs{x} units.\\
\\
Now, we are ready to apply window correlation. Take window size,``w'' as 
input. First window will be from 1st element to w'th element of both the time 
series after aligning by lag ``x''. Find correlation for this window between 
two time-series and save it in an array. Now, slide window by ``w'' elements 
and calculate correlation value again and so on. Now, we have correlation values at 
multiple windows.\\
\\
Now, let's say both the series should have been positively correlated. So, what 
we do is, we choose threshold by MAD test if not provided to us, and find all 
correlation values which are below that threshold and report all those windows 
as anomaly. Similarly, for negative correlation values above the threshold value is reported as anomaly.

\section{Related Functions}

\subsection{correlation(arr1, arr2, maxlag, pos, neg)}

This function calculates correlation between arr1 and arr2 at all possible lags 
between -maxlag to +maxlag, as specified by pos and neg parameters.

\begin{itemize}
 \item Input Parameters
 
 \begin{enumerate}
  \item arr1 \textit{(list)} : Input series 1 as a list of float values
  \item arr2 \textit{(list)} : Input series 2 as a list of float values
  \item maxlag \textit{(int)} : maximum (maxlag) and minimum (-maxlag) lag to 
consider while calculating correlation between arr1 and arr2
  \item pos \textit{(int, 1 or 0)} : To consider positive lag or not, i.e. 1 to 
maxlag, if value is 1, than positive lag will be considered, else not.
  \item neg \textit{(int, 1 or 0)} : To consider negative lag or not, i.e. 
-maxlag to -1, if value is 1, than negative lag will be considered, else not.
 \end{enumerate}

 \item Output \textit{(list)} : \\
  Returns list of tuples of the form \\
  \center{(lag, correlation value at this lag)}
 
\end{itemize}

\subsection{getMaxCorr(arar1,positive\_correlation)}

If both the series are positively correlated than we will be 
interested in maximum positive correlation or if both series are negatively 
correlated than we will be interested in minimum negative correlation, which is 
specified by positive\_correlation parameter.
\\
This function takes list of tuples of the form (lag, correlation value at this 
lag) as  input. Returns lag value at which correlation value is maximum, if 
positive\_correlation is True, and returns lag at which correlation value is 
minimum if positive\_correlation is False. \\



\begin{itemize}
 \item Input Parameters
 
 \begin{enumerate}
  \item arr1 \textit{(list)} : list of tuples of the form \\ (lag, correlation 
value at this lag) \\ i.e. correlation values at various lags
  \item positive\_correlation \textit{(boolean, ``True'' or ``False'')} : 
      \begin{itemize}
       \item True: If value of this parameter is True than it will return lag 
at which correlation value if maximum (positive)
       \item False: If value of this parameter is False than it will return lag 
at which correlation value if minimum (negative)
      \end{itemize}

 \end{enumerate}

 \item Output \textit{(Tuple)} : \\
  returns single tuple of the form (lag,correlation value at this lag), i.e. 
lag at which optimum correlation value is found along with correlation value.
 
\end{itemize}


\subsection{correlationAtLag(series1, series2, lag, window\_size)}

This function first aligns two series by given lag. If lag is positive than it 
shifts start of series2 else start of series1. After aligning both the series 
according to lag, this function calculates correlation between both series at 
all windows. \\
\\
window\_size states size of the window. So, we will start with first window 
taking first ``window\_size'' elements from each series and will calculate 
correlation. We will save this correlation value in list and will slide to next 
window. Next window will start after ``window\_size'' elements. In such a way, we 
calculate, correlation at all windows and return the list of correlation values.\\


\begin{itemize}
 \item Input Parameters
 
 \begin{enumerate}
  \item series1 \textit{(list)} : Input series 1 as a list of float values
  \item series2 \textit{(list)} : Input series 2 as a list of float values
  \item lag \textit{(int)} : lag at which series needs to be adjusted as 
explained above
  \item window\_size \textit{(int)} : window size to be considered
 \end{enumerate}

 \item Output \textit{(list)} : \\
  Returns list of correlation values (of float type) for all windows calculated 
at given lag \\
 
\end{itemize}


\subsection{WindowCorrelationWithConstantLag(arr1, arr2, window\_size,maxlag, 
positive\_correlation, pos, neg)}

This is sort of driver function, which will call above 3 functions. This 
function will first get lag at which series needs to be adjusted. Than using 
this lag, it will calculate correlation values at all windows and will return 
it.

\begin{itemize}
 \item Input Parameters
 
 \begin{enumerate}
  \item arr1 \textit{(list)} : Input series 1 as a list of float values
  \item arr2 \textit{(list)} : Input series 2 as a list of float values
  \item window\_size \textit{(int)} : window size to be considered while 
calculating window correlation
  \item maxlag \textit{(int)} : maximum (maxlag) and minimum (-maxlag) lag to 
consider while calculating correlation between arr1 and arr2, to align both the 
series
  \item positive\_correlation \textit{(boolean, ``True'' or ``False'')} : 
      \begin{itemize}
       \item True: This suggest that both the series are positively correlated
       \item False: This suggest that both the series are negatively correlated
      \end{itemize}
      
  \item pos \textit{(int, 1 or 0)} : If value of this parameter is 1 than we 
will consider positive values for lag, i.e. 1 to +maxlag to align both the 
series initially
  \item neg \textit{(int, 1 or 0)} : If value of this parameter is 1 than we 
will consider negative values for lag, i.e. -maxlag to -1 to align both the 
series initially
  
 \end{enumerate}

 \item Output \textit{(list)} : \\
  Returns tuple of the form (lag,array)
Where lag is lag value for which whole series is shifted and then at that lag, 
we have calculated correlation for all window. Correlation value for all 
windows is stored in array.
 
\end{itemize}

\subsection{anomaliesFromWindowCorrelationWithConstantlag\\(arr1, arr2, 
window\_size=15,maxlag=15, \\positive\_correlation=True, pos=1, neg=1, \\
default\_threshold = True, threshold = 0):}


This is main function of this method. This is driver of whole method. Using 
previously stated methods, it will first gather correlation values at different 
windows. Than depending upon which type of threshold is to be used, it will 
filter out anomalies. If default threshold is to be used, than it will be 
calculated using MAD test on the correlation values at each window, else 
threshold provided by user will be used. \\
\\
Correlation values not satisfying threshold will be reported along with the date 
range of that window.\\


\begin{itemize}
 \item Input Parameters
 
 \begin{enumerate}
  \item arr1 \textit{(list)} : Input series 1 as a list of tuples of the form 
(date,value)
  \item arr2 \textit{(list)} : Input series 2 as a list of tuples of the form 
(date,value)
  \item window\_size \textit{(int)} : window size to be considered while 
calculating window correlation
  \item maxlag \textit{(int)} : maximum (maxlag) and minimum (-maxlag) lag to 
consider while calculating correlation between arr1 and arr2, to align both the 
series
  \item positive\_correlation \textit{(boolean, ``True'' or ``False'')} : 
      \begin{itemize}
       \item True: This suggest that both the series are positively correlated
       \item False: This suggest that both the series are negatively correlated
      \end{itemize}
      
  \item pos \textit{(int, 1 or 0)} : If value of this parameter is 1 than we 
will consider positive values for lag, i.e. 1 to +maxlag to align both the 
series initially
  \item neg \textit{(int, 1 or 0)} : If value of this parameter is 1 than we 
will consider negative values for lag, i.e. -maxlag to -1 to align both the 
series initially
  \item default\_threshold \textit{(boolean, ``True'' or ``False'')} : whether 
to use default threshold or not. If True, default threshold will be used using 
MAD test on calculated correlation values for all windows.
  \item threshold \textit{(float)} : if default\_threshold is False, than this 
user provided threshold will be used.
 \end{enumerate}

 \item Output \textit{(list)} : \\
  This function filter out anomalies and returns them. This function returns
  List of tuples of the form \\ (start\_date,end\_date,correlation\_value), \\
  where (start\_date, end\_date) specifies range of the window and 
correlation\_value is value of correlation of that window
  
\end{itemize}

\section{Description}


Putting all together, here is the summary:\\
\\
Function ''WindowCorrelationWithConstantLag``, first makes use of 
''correlation`` function, to calculate correlation values at all lags to find 
out at which lag it needs to be align. Result of ''correlation`` function is 
passed to ''getMaxCorr`` function. Which will return lag at which optimum value 
of correlation is present. This output will be used by ''correlationAtLag`` 
function, to caluclate correlation at all windows after aligning both series by 
input lag. So, in this way ''WindowCorrelationWithConstantLag`` combines these 
three functions and returns correlation value at each window.
\\
\\
Function ``anomaliesFromWindowCorrelationWithConstantlag`` is the main driver. 
This function calls ''WindowCorrelationWithConstantLag`` and gets the 
correlation values at all windows and filters out anomalies (either using 
threshold calculated by MAD test or by user provided threshold) and returns them 
in the format of (start\_date,end\_date,correlation\_value),
  where (start\_date, end\_date) specifies range of the window and 
correlation\_value is value of correlation of that window.